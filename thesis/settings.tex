\PassOptionsToPackage{table,svgnames,dvipsnames}{xcolor}

\usepackage[utf8]{inputenc}
\usepackage[T1]{fontenc}
\usepackage[ngerman,english]{babel}
\usepackage[autostyle]{csquotes}
\usepackage[%
  backend=bibtex,
  url=false,
  %style=alphabetic,
  %maxnames=4,
  %minnames=3,
  maxbibnames=99,
  %giveninits,
  uniquename=init]{biblatex} % TODO: adapt citation style
\usepackage{graphicx}
\usepackage{scrhack} % necessary for listings package
\usepackage{tikz}
\usepackage{pgfplots}
\usepackage{pgfplotstable}
\usepackage{booktabs}
\usepackage[final]{microtype}
\usepackage{caption}
\usepackage[hidelinks]{hyperref} % hidelinks removes colored boxes around references and links
\hypersetup{
 colorlinks,
 linktoc=page,
 linkcolor=blue,
 citecolor=blue,
 urlcolor=blue
}

\bibliography{bibliography}

\setkomafont{disposition}{\normalfont\bfseries} % set font

% Add table of contents to PDF bookmarks
\BeforeTOCHead[toc]{{\cleardoublepage\pdfbookmark[0]{\contentsname}{toc}}}

% Settings added by Kevin
\usepackage{enumerate}
\usepackage{multicol}

% Math packages
\usepackage{bm}
\usepackage{mathtools}
\usepackage{amsmath}
\usepackage{amssymb}

% Proof system
\usepackage{amsthm}
% spacing fixes for proof system
\makeatletter
\def\thm@space@setup{%
  \thm@preskip=\parskip \thm@postskip=0pt
}
\makeatother
\theoremstyle{plain}
\newtheorem{thm}{Theorem}[chapter]
\newtheorem{lem}[thm]{Lemma}
\newtheorem{prop}[thm]{Proposition}
\newtheorem{cor}[thm]{Corollary}
\theoremstyle{definition}
\newtheorem{defn}[thm]{Definition}
\newtheorem{exmpl}[thm]{Example}
%\newtheoremstyle{rem} % name
    %{\topsep}                    % Space above
    %{\topsep}                    % Space below
    %{}                   % Body font
    %{}                           % Indent amount
    %{\bf}                   % Theorem head font
    %{:}                          % Punctuation after theorem head
    %{.5em}                       % Space after theorem head
    %{}  % Theorem head spec (can be left empty, meaning ‘normal’)
%\theoremstyle{rem}
%\newtheorem*{remark}{Note}

% Algorithms
\input{settings_algorithms}

% Graphs
\usetikzlibrary{calc,arrows.meta,positioning}
\usepackage{tikz-3dplot}
\usepackage{subfig}

% Notes
\usepackage{xargs} % Use more than one optional parameter in a new commands
\usepackage[colorinlistoftodos,prependcaption,textsize=tiny]{todonotes}

% Custom commands
\DeclareMathOperator*{\argmin}{arg\,min}
\DeclareMathOperator*{\argmax}{arg\,max}
\newcommand{\mx}{\mathcal{X}}
\newcommand{\inp}{\mathcal{I}}
\newcommand{\costs}{c}
\newcommand{\opcosts}{c_{op}}
\newcommand{\swcosts}{c_{sw}}
\newcommand{\fromto}[2]{\{#1,\ldots,#2\}}
\newcommand{\dotcup}{\mathbin{\mathaccent\cdot\cup}}
\newcommandx{\unsure}[2][1=]{\todo[linecolor=red,backgroundcolor=red!25,bordercolor=red,#1]{#2}}
%\newcommandx{\unsure}[2][1=]{}

% Line breaks after paragraph
\usepackage[parfill]{parskip}
% double line spacing
%\linespread{1.5}
