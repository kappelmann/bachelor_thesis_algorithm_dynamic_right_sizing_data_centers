% !TeX root = ../main.tex
% Add the above to each chapter to make compiling the PDF easier in some editors.

\chapter{Preliminaries}\label{chap:preliminaries}
In this chapter, we conduct the preparatory work that will lay the foundations for our algorithms. For this, we will analyze the structure of feasible schedules to find characteristics of optimal strategies. These characteristics will then allow us to greatly simplify our optimization conditions.

We begin by examining the state sequences of feasible schedules. As we consider homogeneous servers, we do not care which exact machines process the given work loads. Rather, we only care about the number of active servers and the distribution of loads between them. It is in particular unreasonable to shut down a machine and to power on a different one in return; we could just keep the first server powered on and thereby save switching costs.
This investigation is captured by our first proposition.
\begin{prop}[Reasonable switching]\label{prop:reasonable_switching}
Given a problem instance $\inp$ and a feasible schedule $\Sigma$, there exists a feasible schedule $\Sigma'$ such that 
\begin{enumerate}[label=(\roman*)]
		\item $\costs(\Sigma')\le \costs(\Sigma)$ and 
		\item $\Sigma'$ never powers on and shuts down servers at the same time slot. More formally, $\Sigma'$ satisfies the formula $\forall t\in[T]:F$ where $F$ is defined as
\begin{equation}
	F\coloneqq\bigl(\forall i\in[m]:s_{i,t}-s_{i,t-1}\ge0\bigr)\lor\bigl(\forall i\in[m]:s_{i,t}-s_{i,t-1}\le 0\bigr)\label{eq:reasonable_switching}
\end{equation}\label{itm:cond_reasonable_switching}
\vspace{-\baselineskip}\vspace{-\belowdisplayskip}
\end{enumerate}
\end{prop}
\begin{proof}
Let $\Sigma=(\mathcal{S},\mathcal{L})$ be a feasible schedule for $\inp$. We give a procedure that repeatedly modifies $\Sigma$ such that it satisfies condition~\ref{itm:cond_reasonable_switching} and reduces or retains its cost.
	
Let $t\in[T]$ be the first time slot that falsifies~\eqref{eq:reasonable_switching}. If there does not exist such a time slot, we are finished. Otherwise, we can obtain machines $i,j\in[m]$ such that $s_{i,t}-s_{i,t-1}=1$ and \makebox{$s_{j,t}-s_{j,t-1}=-1$}, that is server $i$ powers on at time $t$ and server $j$ shuts down. Without loss of generality, we may assume $i<j$. Since all servers are sleeping at time $t=0$, we have
\begin{equation*}
	s_{k,1}-s_{k,0}=s_{k,1}-0=s_{k,1}\ge 0,\quad\forall k\in[m]
\end{equation*}
which shows that formula~\eqref{eq:reasonable_switching} is satisfied for $t=1$. Thus, we further assume $t>1$. Now consider the state sequences of server $i$ and $j$:
\begin{align*}
	S_i&=(s_{i,1},\dotsc,s_{i,t-1}=0,s_{i,t}=1,\dotsc,s_{i,T})\\
	S_j&=(s_{j,1},\dotsc,s_{j,t-1}=1,s_{j,t}=0,\dotsc,s_{j,T})
\end{align*}
We modify $S_i$ and $S_j$ by swapping their states for time slots $\ge t$, i.e.\ we set
\begin{align*}
	S_i'&\coloneqq(s_{i,1},\dotsc,s_{i,t-1}=0,s_{j,t}=0,\dotsc,s_{j,T})\\
	S_j'&\coloneqq(s_{j,1},\dotsc,s_{j,t-1}=1,s_{i,t}=1,\dotsc,s_{i,T})
\end{align*}
Similarly, we need to swap the assigned loads for server $i$ and $j$:
\begin{align*}
	L_i'&\coloneqq(\lambda_{i,1},\dotsc,\lambda_{i,t-1},\lambda_{j,t},\dotsc,\lambda_{j,T})\\
	L_j'&\coloneqq(\lambda_{j,1},\dotsc,\lambda_{j,t-1},\lambda_{i,t},\dotsc,\lambda_{i,T})
\end{align*}
Finally, we construct a new schedule $\Sigma'\coloneqq(\mathcal{S}',\mathcal{L}')$ by setting
\begin{align*}
	\mathcal{S}'&\coloneqq\bigl(S_1,\dotsc,S_{i-1},S_i',S_{i+1},\dotsc,S_{j-1},S_j',S_{j+1},\dotsc,S_T\bigr)\\
	\mathcal{L}'&\coloneqq\bigl(L_1,\dotsc,L_{i-1},L_i',L_{i+1},\dotsc,L_{j-1},L_j',L_{j+1},\dotsc,L_T\bigr)
\end{align*}
We now want to verify that $\Sigma'$ is a feasible schedule, that is $\Sigma'$ satisfies~\eqref{eq:feasible_constraint}. For time slots $<t$, the schedules $\Sigma'$ and $\Sigma$ still coincide. For time slots $\ge t$, we only changed the order of summation in~\eqref{eq:feasible_constraint}. Thus, $\Sigma'$ is feasible.

Further, $\Sigma$ and $\Sigma'$ coincide in their operating costs; however, by exchanging $\mathcal{S}$ with $\mathcal{S}'$, we reduced the number of servers powering up at time $t$. As we assume $\beta\ge0$, we conclude $\costs(\Sigma')\le \costs(\Sigma)$. Moreover, we decreased the number of servers violating~\eqref{eq:reasonable_switching} at time $t$. Hence, by repeating the described process on $\Sigma'$, we obtain a terminating procedure that returns a schedule satisfying the conditions.
\end{proof}

Our next proposition -- which will pose the cornerstone of our subsequent works -- requires the use of Jensen's inequality, a well-known and frequently used analytic result found by Johan Jensen in 1906. It generalizes the idea that the secant line of a convex function lies above the graph of the function; more specifically, it states that the value of a convex function at a finite convex-combination of sampling points is less than or equal to the convex-combination of the function values at the sampling points.
\begin{lem}[Jensen's inequality]\label{lem:jensens-inequality}
Let $f:D\rightarrow\mathbb{R}$ be a convex function, $n\in\mathbb{N}$, \makebox{$\lambda_1,\dotsc,\lambda_n\in D$}, and $\mu_1,\dotsc,\mu_n\in[0,1]$ satisfying $\sum\limits_{i=1}^{n}\mu_i=1$. Then the following inequality holds:
\begin{equation*}
	f\left(\sum_{i=1}^n(\mu_i \lambda_i)\right) \le \sum_{i=1}^n\bigl(\mu_i f(\lambda_i)\bigr)
\end{equation*}
\end{lem}
\begin{proof}
We proof the claim by induction on $n\in\mathbb{N}$.
\begin{itemize}
\item\underline{Basis:} For $n=1$, we have $\mu_1=1$ and thus
\begin{equation*}
	f\left(\sum\limits_{i=1}^1(\mu_i\lambda_i)\right)=f(\lambda_1)=\sum\limits_{i=1}^1\bigl(\mu_if(\lambda_i)\bigr)
\end{equation*}
\item\underline{Step:} Let $n\in\mathbb{N}$ be arbitrary and fixed.
\begin{itemize}
	\item\underline{I.H.:} The assertion holds for $n$.
	\item\underline{Claim:} The assertion holds for $n+1$.
	\item\underline{Proof:} Since $\sum\limits_{i=1}^{n+1}\mu_i=1$, $\mu_i\in[0,1]$, and $n+1\ge 2$, at least one $\mu_i$ must be smaller than 1. Without loss of generality, we may assume $\mu_1<1$.
	\begin{equation*}
		f\left(\sum\limits_{i=1}^{n+1}(\mu_i\lambda_i)\right)=f\left(\mu_1\lambda_1+\sum\limits_{i=2}^{n+1}(\mu_i\lambda_i)\right)\stackrel{\mu_1\neq1}{=}f\left(\mu_1\lambda_1+(1-\mu_1)\sum\limits_{i=2}^{n+1}\frac{\mu_i\lambda_i}{1-\mu_1}\right)
	\end{equation*}
	As $f$ is convex, we have
	\begin{equation*}
		f\left(\mu_1\lambda_1+(1-\mu_1)\sum\limits_{i=2}^{n+1}\frac{\mu_i\lambda_i}{1-\mu_1}\right)\stackrel{f\text{ convex~\eqref{eq:convex_condition}}}{\le} \mu_1f(\lambda_1)+(1-\mu_1)f\left(\sum\limits_{i=2}^{n+1}\frac{\mu_i\lambda_i}{1-\mu_1}\right)
	\end{equation*}
	Since $\sum\limits_{i=2}^{n+1}\frac{\mu_i}{1-\mu_1}=1$, we can apply our induction hypothesis.
	\begin{equation*}
		\mu_1f(\lambda_1)+(1-\mu_1)f\left(\sum\limits_{i=2}^{n+1}\frac{\mu_i\lambda_i}{1-\mu_1}\right)\stackrel{\text{I.H.}}{\le} \mu_1f(\lambda_1)+(1-\mu_1)\sum\limits_{i=2}^{n+1}\left(\frac{\mu_i}{1-\mu_1}f(\lambda_i)\right)
	\end{equation*}
	We combine our steps and obtain
	\begin{align*}
		f\left(\sum\limits_{i=1}^{n+1}(\mu_i\lambda_i)\right)&\le \mu_1f(\lambda_1)+(1-\mu_1)\sum\limits_{i=2}^{n+1}\left(\frac{\mu_i}{1-\mu_1}f(\lambda_i)\right)\\
		&=\mu_1f(\lambda_1)+\sum\limits_{i=2}^{n+1}\bigl(\mu_if(\lambda_i)\bigr)=\sum_{i=1}^{n+1}\bigl(\mu_i f(\lambda_i)\bigr)
	\end{align*}
\end{itemize}
\end{itemize}
Thus, the assertion holds for any natural number $n$.
\end{proof}
Next, we want to consider the sequence of the number of active servers $\mx$ defined as
\begin{flalign*}
	&&\mx\coloneqq(x_1,\dotsc,x_T)\in[m]_0^T&&\text{where}&&x_t\coloneqq\sum\limits_{i=1}^{m}s_{i,t}\in[m]_0&&&
\end{flalign*}	
As we assume that all machines are sleeping at times $t\notin[T]$, we have $x_t=0$ for $t\notin[T]$. We now want to establish an optimal scheduling strategy given a fixed number of active servers. It turns out that an even load distribution seems a very desirable strategy.\begin{prop}[Even load distribution]\label{prop:even_load_distribution}
Given $x_t\in\mathbb{N}$ active servers in time slot $t$, an arrival rate $\lambda_t\in[0,x_t]$, and a convex operating cost function $f$, a most cost-efficient and feasible scheduling strategy is to assign each active server a load of $\lambda_t/x_t$.
\end{prop}
\begin{proof}
Let $\Sigma$ be an arbitrary feasible schedule using $x_t$ servers in time slot $t$, and let $A$ be its set of active servers in time slot $t$, that is $A\coloneqq\{i\in[m]\mid s_{i,t}=1\}$.
Consider the operating costs of $\Sigma$ at time $t$ given by
\begin{equation*}
	\sum\limits_{i=1}^{m}\bigl(f(\lambda_{i,t})\cdot s_{i,t}\bigr)=\sum\limits_{i\in A}\bigl(f(\lambda_{i,t})\cdot1\bigr)+\sum\limits_{i\in [m]\setminus A}\bigl(f(\lambda_{i,t})\cdot0\bigr)=\sum\limits_{i\in A}f(\lambda_{i,t})
\end{equation*}
Since $\Sigma$ is feasible (see constraint~\eqref{eq:feasible_constraint}), we have 
\begin{equation*}
	\sum\limits_{i\in A}\lambda_{i,t}=\lambda_t
\end{equation*}
Hence, we can obtain weights $\mu_1,\dotsc,\mu_{x_t}\in[0,1]$ that relate $\lambda_t$ to $\lambda_{i,t}$ for $i\in A$ such that
\begin{align}
	\sum\limits_{j=1}^{x_t}\mu_j=1\qquad\text{and}\qquad \sum\limits_{i\in A}f(\lambda_{i,t})=\sum\limits_{j=1}^{x_t}f(\mu_j\lambda_t)\label{eq:mu_lambda_costs}
\end{align}
In particular, we have 
\begin{equation}
	\sum_{j=1}^{x_t}\mu_j\lambda_t=\lambda_t\label{eq:mu_convex_comb}
\end{equation}
Using these weights, we now consider the operating costs of a schedule $\Sigma^*$ that evenly distributes $\lambda_t$ to its $x_t$ active servers:
\begin{align*}
	\sum\limits_{j=1}^{x_t}f\left(\frac{\lambda_t}{x_t}\right)=x_tf\left(\frac{\lambda_t}{x_t}\right)\stackrel{\eqref{eq:mu_convex_comb}}{=}x_tf\left(\sum\limits_{j=1}^{x_t}{\frac{\mu_j\lambda_t}{x_t}}\right)
\end{align*}
With the fact that $\sum_{j=1}^{x_t}(1/x_t)=1$ and the use of Jensen's inequality (Lemma~\ref{lem:jensens-inequality}), we can give an upper bound for the costs:
\begin{align*}
	x_tf\left(\frac{\lambda_t}{x_t}\right)\stackrel{\text{Lemma~\ref{lem:jensens-inequality}}}{\le}x_t\sum\limits_{j=1}^{x_t}\left(\frac{1}{x_t}f(\mu_j\lambda_t)\right)=\frac{x_t}{x_t}\sum\limits_{j=1}^{x_t}f(\mu_j\lambda_t)=\sum\limits_{j=1}^{x_t}f(\mu_j\lambda_t)\stackrel{\eqref{eq:mu_lambda_costs}}{=}\sum\limits_{i\in A}f(\lambda_{i,t})
\end{align*}
Thus, the operating costs of $\Sigma^*$ give a lower bound for the operating costs of $\Sigma$, and the claim follows.
\end{proof}
As a special case, we can apply our just derived proposition to optimal schedules.
\begin{cor}\label{cor:even_load_distribution}
Given a problem instance $\inp$, there exists an optimal schedule $\Sigma^*$ that evenly distributes its arrival rates to its active servers in each time slot.
\end{cor}
\begin{proof}
Let $\Sigma=(\mathcal{S},\mathcal{L})$ be an optimal schedule for $\inp$. We exchange $\mathcal{L}$ with a new strategy~$\mathcal{L}^*$ that evenly distributes the arrival rates to all active servers of $\Sigma$ in each time slot, that is we set $\Sigma^*\coloneqq(\mathcal{S},\mathcal{L}^*)$. By Proposition~\ref{prop:even_load_distribution}, we have $\costs(\Sigma^*)\le\costs(\Sigma)$ and, since $\Sigma$ is optimal, $\costs(\Sigma)\le\costs(\Sigma^*)$. Thus, we conclude $\costs(\Sigma^*)=\costs(\Sigma)$, i.e.\ $\Sigma^*$ is optimal, which finishes the proof.
\end{proof}
As a result of Corollary~\ref{cor:even_load_distribution}, we can restrict ourselves to finding an optimal schedule that evenly distributes its arrival rates to its active servers. We now combine our results to derive the main theorem of our preliminary work.
\begin{thm}\label{thm:even_load_distribution_reasonable_switching}
Given a problem instance $\inp$, there exists an optimal schedule $\Sigma^*$ that evenly distributes its arrival rates and never powers on and shuts down servers at the same time slot.
\end{thm}
\begin{proof}
By Corollary~\ref{cor:even_load_distribution}, we obtain an optimal schedule $\Sigma$ that evenly distributes its arrival rates to its active servers. Applying the procedure given in Proposition~\ref{prop:reasonable_switching} to $\Sigma$ yields~$\Sigma^*$ that further satisfies~\eqref{eq:reasonable_switching} for all $t\in[T]$ and $\costs(\Sigma^*)\le\costs(\Sigma)$. Since $\Sigma$ is optimal, we conclude $\costs(\Sigma^*)=\costs(\Sigma)$, and the claim follows.
\end{proof}

Theorem~\ref{thm:even_load_distribution_reasonable_switching} allows us to identify an optimal schedule by its sequence of the number of active servers $\mx$ and thereby to simplify our optimization conditions~\eqref{eq:schedule_costs} and \eqref{eq:feasible_constraint}. 
For this, given a problem instance $\inp$, we define the operating cost function $\opcosts(x,\lambda)$, which describes the costs incurred by evenly distributing $\lambda$ on $x$ active servers using $f$:
\begin{equation}\label{eq:mx_schedule_op_costs}
	\opcosts:[m]_0\times[0,m]\rightarrow\mathbb{R}\dotcup\{\infty\},\quad \opcosts(x,\lambda)=\begin{cases}
	  \infty, & \text{if $\lambda>x$}\\
	  xf(\lambda/x), & \text{if $x\ne 0\land\lambda\le x$}\\
          0, & \text{if $x=\lambda=0$}
	  \end{cases}
\end{equation}
We assign infinite costs in case $\lambda>x$ as there would be too few active servers to process the arrival rate, i.e.\ the schedule would not be feasible. Naturally, this definition can be lifted to whole schedules by setting
\begin{equation}\label{eq:mx_schedule_op_costs_complete}
	\opcosts(\mx)\coloneqq\sum\limits_{i=1}^T\opcosts(x_i,\lambda_i)
\end{equation}
which will come in handy in later sections. Next, we define the switching costs function $\swcosts(x_{t-1},x_t)$, which describes the incurred costs when changing the number of active server from $x_{t-1}$ to $x_t$:
\begin{equation}\label{eq:mx_schedule_sw_costs}
	\swcosts(x_{t-1},x_t)\coloneqq\beta\max\{0,x_t-x_{t-1}\}
\end{equation}
Lastly, we can define the cost function $\costs(x_{t-1},x_t,\lambda_t)$, which describes the incurring costs for a single time step using an even distribution of loads:
\begin{equation}\label{eq:mx_schedule_costs}
	\costs(x_{t-1},x_{t},\lambda_t)\coloneqq\opcosts(x_t,\lambda_t)+\swcosts(x_{t-1},x_t)
\end{equation}
The optimization conditions for a schedule now simplify to one single minimization:
\begin{align}
	\underset{\mx}{\text{minimize}}\quad \costs(\mx)\coloneqq\sum\limits_{t=1}^{T}\costs(x_{t-1},x_{t},\lambda_t)\label{eq:mx_schedule_total_costs}
\end{align}
We subsequently call a schedule $\mx$ \emph{optimal} if it satisfies~\eqref{eq:mx_schedule_total_costs}. Having greatly simplified our optimization conditions, we next approach the first main goal of our work: the development of an optimal offline algorithm.
