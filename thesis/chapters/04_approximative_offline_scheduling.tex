% !TeX root = ../main.tex
% Add the above to each chapter to make compiling the PDF easier in some editors.

\chapter{Approximative Offline Scheduling}
Heretofore, we have derived two optimal offline algorithms for our scheduling problem; unfortunately, the runtime complexities of both algorithms are exponential in the input size of the number of servers $m$. Needless to say, we want to reduce this exponential runtime. For this, we must slightly loosen our aspirations, that is we move to approximative methods. Further, as we will see in the following sections, we need to assume that our convex operating costs function $f$ is monotonically increasing; however, this restriction is of no great significance since in practice most operating costs functions fulfill this requirement.\unsure{Any reference?} 

In this chapter, we will first modify our algorithm derived in Section~\ref{sec:opt_offline_pseudo_lin} to obtain a 4-optimal offline algorithm with linear runtime complexity. As a next step, we will generalize our first approach to allow for arbitrary $1+\beps$-approximations with TODO complexity.

\section{A 4-Optimal Algorithm}
Recall our algorithm and its corresponding graph $G$ derived in Section~\ref{sec:opt_offline_pseudo_lin}. The algorithm's runtime complexity of $\Theta(Tm)$ is determined by the number of nodes and edges of $G$. Hence, as we desire to reduce our runtime complexity, we need to reduce the number of nodes and edges in $G$. In particular, we must get rid of the factor $m$. This factor is a consequence of the ``height'' of our graph, that is the number of nodes in each layer. Therefore, we have to ``thin out'' $G$ by reducing its number of nodes in each layer.

As we saw in Equation~\eqref{eq:inp_size}, the size of our input $\inp$ is given by $\mathcal{O}\bigl(T\log_2(m)+\log_2(\beta)\bigr)$. Consequently, in order to obtain a linear runtime complexity, we may reduce the graph's height from $m+1$ to a logarithmic one of $\log(m)$. Given this observation, it seems intuitive for a computer scientist to choose a logarithmic scale for the number of servers in each layer. More formally, given a problem instance $\inp$, we consider the following graph:
\begin{align*}
	b&\coloneqq\lfloor\log_2(m)\rfloor\\
	B&\coloneqq\{0,2^0,2^1,\ldots,2^b,m\}\\
	V&\coloneqq\bigl\{v_{x,t\downarrow}\mid x\in B,t\in[T]\bigr\}\dotcup\bigl\{v_{x,t\uparrow}\mid x\in B, t\in[T-1]\bigr\}\dotcup\{v_{0,0}\}\\
	E_s&\coloneqq\bigl\{(v_{0,0},v_{x,1\downarrow})\mid x\in B\bigr\}\\
	E_\downarrow&\coloneqq\bigl\{(v_{2^i,t\downarrow},v_{2^{i-1},t\downarrow})\mid i\in[b],t\in[T]\bigr\}\dotcup\bigl\{(v_{2^0,t\downarrow},v_{0,t\downarrow})\mid t\in[T]\bigr\}\dotcup\\
	&\phantom{{}\coloneqq{}}\bigl\{(v_{m,t\downarrow},v_{2^b,t\downarrow})\mid t\in[T]\bigr\}\\
	E_\uparrow&\coloneqq\bigl\{(v_{2^{i-1},t\uparrow},v_{2^i,t\uparrow})\mid i\in[b],t\in[T-1]\bigr\}\dotcup\bigl\{(v_{0,t\uparrow},v_{2^0,t\uparrow})\mid t\in[T-1]\bigr\}\dotcup\\
	&\phantom{{}\coloneqq{}}\bigl\{(v_{2^b,t\uparrow},v_{m,t\uparrow})\mid t\in[T-1]\bigr\}\\
E_{\downarrow\uparrow}&\coloneqq\bigl\{(v_{x,t\downarrow},v_{x,t\uparrow})\mid x\in N,t\in[T-1]\bigr\}\\
	E_{\uparrow\downarrow}&\coloneqq\bigl\{(v_{x,t\uparrow},v_{x,t+1\downarrow})\mid x\in N,t\in[T-1]\bigr\}\\
	E&\coloneqq E_s\dotcup E_\downarrow\dotcup E_\uparrow\dotcup E_{\downarrow\uparrow}\dotcup E_{\uparrow\downarrow}\\
	c_G(e)&\coloneqq
	\begin{cases}
		\costs(0,x,\lambda_1), & \text{if $e=(v_{0,0},v_{x,1\downarrow})\in E_s$}\\
		\opcosts(x,\lambda_{t+1}), & \text{if $e=(v_{x,t\uparrow},v_{x,t+1\downarrow})\in E_{\uparrow\downarrow}$}\\
		(x'-x)\beta, & \text{if $e=(v_{x,t\uparrow,v_{x',t\uparrow}})\in E_\uparrow$}\\
		0, & \text{if $e\in(E_\downarrow\dotcup E_{\downarrow\uparrow})$}
	\end{cases}\\
	G&\coloneqq(V,E,c_G)
\end{align*}
A more appealing, graphical representation can be found in the following figure.
\begin{figure}[H]
\centering
\resizebox{\textwidth}{!}{
\begin{tikzpicture}[->,>=stealth',auto,node distance=2.2cm,thick,node/.style={minimum size=1.5cm,circle,draw}]
  \node[node] (1) {0,0};
  \node[node] (4) [below right=4cm of 1] {0,1$\downarrow$};
  \node[node] (3) [above =0.5cm of 4] {$2^0$,1$\downarrow$};
  \node[node] (2) [above right=4cm of 1] {m,1$\downarrow$};
  \node[node] (5) [below =0.5cm of 2] {$2^b$,1$\downarrow$};
  \node[node] (7) [right =of 3] {$2^0$,1$\uparrow$};
  \node[node] (6) [right =of 2] {m,1$\uparrow$};
  \node[node] (8) [right =of 4] {0,1$\uparrow$};
  \node[node] (9) [right =of 5] {$2^b$,1$\uparrow$};
  \node[node] (11) [right =of 7] {$2^0$,2$\downarrow$};
  \node[node] (10) [right =of 6] {m,2$\downarrow$};
  \node[node] (12) [right =of 8] {0,2$\downarrow$};
  \node[node] (13) [right =of 9] {$2^b$,2$\downarrow$};
  \node[node] (15) [right =of 11] {$2^0$,2$\uparrow$};
  \node[node] (14) [right =of 10] {m,2$\uparrow$};
  \node[node] (16) [right =of 12] {0,2$\uparrow$};
  \node[node] (17) [right =of 13] {$2^b$,2$\uparrow$};
  \node[node] (19) [right =5cm of 15] {$2^0$,T$\downarrow$};
  \node[node] (18) [right =5cm of 14] {m,T$\downarrow$};
  \node[node] (20) [right =5cm of 16] {0,T$\downarrow$};
  \node[node] (21) [right =5cm of 17] {$2^b$,T$\downarrow$};

  \node (22) at ($(2)!.5!(4)$) {};
  \node (23) at ($(2)!.487!(4)$) {\vdots};
  \node (24) at ($(6)!.487!(8)$) {\vdots};
  \node (25) at ($(10)!.487!(12)$) {\vdots};
  \node (26) at ($(14)!.487!(16)$) {\vdots};
  \node at ($(18)!.487!(20)$) {\vdots};
  \node at ($(23)!.487!(24)$) {\vdots};
  \node at ($(24)!.487!(25)$) {\vdots};
  \node at ($(25)!.487!(26)$) {\vdots};

  \node (27) at ($(14)!.5!(18)$) {\ldots};
  \node at ($(15)!.5!(19)$) {\ldots};
  \node (28) at ($(16)!.5!(20)$) {\ldots};
  \node at ($(17)!.5!(21)$) {\ldots};
  \node at ($(27)!.5!(28)$) {\ldots};

  \path[every node/.style={font=\sffamily\small}]
    (1) edge[red] node[black,above left] {$\costs(0,m,\lambda_1)$} (2)
	edge node[label={[xshift=0.8cm, yshift=-0.9cm]$\costs(0,2^b,\lambda_1)$}] {} (5)
	edge node[above right=-0.13cm] {$\costs(0,2^0,\lambda_1)$} (3)
	edge node[below left] {$\costs(0,0,\lambda_1)$} (4)
    (1) edge ($(1)!.82!(22)$)

    (2) edge node[above] {$0$} (6)
    (3) edge node[above] {$0$} (7)
    (4) edge[red] node[black,above] {$0$} (8)
    (5) edge node[above] {$0$} (9)

    (2) edge[red] node[black,right] {$0$} (5)
    (3) edge[red] node[black,right] {$0$} (4)

    (9) edge[red] node[black,right] {$(m-2^b)\beta$} (6)
    (8) edge[red] node[black,right] {$\beta$} (7)

    (6) edge[red] node[black,above] {$\opcosts(m,\lambda_2)$} (10)
    (7) edge node[above] {$\opcosts(2^0,\lambda_2)$} (11)
    (8) edge node[above] {$\opcosts(0,\lambda_2)$} (12)
    (9) edge node[above] {$\opcosts(2^b,\lambda_2)$} (13)

    (10) edge node[above] {$0$} (14)
    (11) edge node[above] {$0$} (15)
    (12) edge[red] node[black,above] {$0$} (16)
    (13) edge node[above] {$0$} (17)

    (10) edge[red] node[black,right] {$0$} (13)
    (11) edge[red] node[black,right] {$0$} (12)

    (17) edge[red] node[black,right] {$(m-2^b)\beta$} (14)
    (16) edge[red] node[black,right] {$\beta$} (15)

    (5) edge[red] node[black,right] {$0$} ($(5)!.35!(3)$)
    ($(5)!.65!(3)$) edge[red] node[black,right] {$0$} (3)

    ($(7)!.65!(9)$) edge[red] node[black,right] {$2^{b-1}\beta$} (9)
    (7) edge[red] node[black,right] {$2^0\beta$} ($(7)!.35!(9)$)
 
    (13) edge[red] node[black,right] {$0$} ($(13)!.35!(11)$)
    ($(13)!.65!(11)$) edge[red] node[black,right] {$0$} (11)

    ($(15)!.65!(17)$) edge[red] node[black,right] {$2^{b-1}\beta$} (17)
    (15) edge[red] node[black,right] {$2^0\beta$} ($(15)!.35!(17)$)


    (14) edge[red] node[label={[black,xshift=0cm, yshift=-0.26cm]$\opcosts(m,\lambda_3)$}] {} ($(14)!.4!(18)$)
    (15) edge node[label={[xshift=0cm, yshift=-0.26cm]$\opcosts(2^0,\lambda_3)$}] {} ($(15)!.4!(19)$)
    (16) edge node[label={[xshift=0cm, yshift=-0.26cm]$\opcosts(0,\lambda_3)$}] {} ($(16)!.4!(20)$)
    (17) edge node[label={[xshift=0cm, yshift=-0.26cm]$\opcosts(2^b,\lambda_3)$}] {} ($(17)!.4!(21)$)

    ($(14)!.6!(18)$) edge[red] node[black,above] {$\opcosts(m,\lambda_T)$} (18)
    ($(15)!.6!(19)$) edge node[above] {$\opcosts(2^0,\lambda_T)$} (19)
    ($(16)!.6!(20)$) edge node[above] {$\opcosts(0,\lambda_T)$} (20)
    ($(17)!.6!(21)$) edge node[above] {$\opcosts(2^b,\lambda_T)$} (21)

    (18) edge[red] node[black,right] {$0$} (21)
    (19) edge[red] node[black,right] {$0$} (20)
    (21) edge[red] node[black,right] {$0$} ($(21)!.35!(19)$)
    ($(21)!.65!(19)$) edge[red] node[black,right] {$0$} (19);
\end{tikzpicture}
}
\caption{Graph for a linear, 4-optimal offline algorithm; the path of the topological sorting is highlighted in red.}
\label{fig:graph_pseudo_lin}
\end{figure}

\section{A $1+\beps$-Optimal Algorithm}
