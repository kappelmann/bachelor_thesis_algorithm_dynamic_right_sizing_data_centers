% !TeX root = ../main.tex
% Add the above to each chapter to make compiling the PDF easier in some editors.

\chapter{Optimal offline scheduling}
In this section, we derive two optimal offline algorithms based on our preliminary work.
First, we reduce our problem specified by $\inp$ to a shortest path problem of a level structured graph $G$. We then proceed to find a shortest path in $G$ and thereby an optimal schedule for $\inp$ in pseudo-polynomial time $\Theta(Tm^2)$.
After that, we refine our initial approach to derive an improved algorithm with pseudo-linear complexity $\Theta(Tm)$.

\section{A pseudo-polynomial algorithm}\label{sec_opt_offline_pseudo_poly}
Let $\inp$ be a problem instance. Thanks to our preliminary work, we know that there exists an optimal schedule which is identifiable by its sequence of active servers $\mx$. In order to find this sequence $\mx$, consider the weighted, level structured graph $G$ defined as follows:
\begin{align*}
	&V\coloneqq\bigl\{v_{x,t}\mid x\in\fromto{0}{m},t\in[T]\bigr\}\dotcup\{v_{0,0},v_{0,T+1}\}\\
	&E\coloneqq\bigl\{(v_{x,t},v_{x',t+1})\mid x,x'\in\fromto{0}{m},t\in\fromto{0}{T},v_{x,t},v_{x',t+1}\in V\bigr\}\\
	&c_G(v_{x,t},v_{x',t+1})\coloneqq\costs(x,x',\lambda_{t+1})\\
	&G\coloneqq(V,E,c_G)
\end{align*}
For any possible amount of active servers $x$ and any time slot $t$, we add a node $v_{x,t}$. Moreover, we add a start node $v_{0,0}$ as well as an end node $v_{0,T+1}$. Next, we connect all nodes to their successors with respect to time. Semantically, $v_{x,t}$ denotes the state of distributing the arrival rate $\lambda_{t}$ evenly to $x$ servers in time slot $t$. For any edge connecting $v_{x,t}$ with $v_{x',t+1}$, we assign costs $\costs(x,x',\lambda_{t+1})$, i.e.\ the cost of taking the edge corresponds to switching from $x$ to $x'$ machines and processing the load $\lambda_{t+1}$ with $x'$ machines. A graphical representation can be found in the following figure.
\begin{figure}[H]
\centering
\scalebox{0.86}{
\begin{tikzpicture}[->,>=stealth',auto,node distance=3cm,thick,node/.style={minimum size=1.2cm,circle,draw}]

  \node[node] (1) {0,0};
  \node[node] (4) [below right =of 1] {0,1};
  \node[node] (3) [above =0.5cm of 4] {1,1};
  \node[node] (2) [above right=of 1] {m,1};
  \node[node] (6) [right =of 3] {1,2};
  \node[node] (5) [right =of 2] {m,2};
  \node[node] (7) [right =of 4] {0,2};
  \node[node] (9) [right =of 6] {1,T};
  \node[node] (8) [right =of 5] {m,T};
  \node[node] (10) [right =of 7] {0,T};
  \node[node] (11) [above right =of 10] {0,T+1};

  \node at ($(5)!.4!(8)$) {\ldots};
  \node (12) at ($(6)!.4!(9)$) {\ldots};
  \node at ($(7)!.4!(10)$) {\ldots};
  \node [above=1.72cm of 12]{\ldots};

  \node at ($(2)!.5!(3)$) {\vdots};
  \node at ($(2)!.5!(6)$) {\vdots};
  \node at ($(5)!.5!(6)$) {\vdots};
  \node at ($(8)!.5!(9)$) {\vdots};

  \path[every node/.style={font=\sffamily\small}]
    (1) edge node[above left] {$\costs(0,m,\lambda_1)$} (2)
	edge node[above right=-0.15cm] {$\costs(0,1,\lambda_1)$} (3)
	edge node[below left] {$\costs(0,0,\lambda_1)$} (4)
    (2) edge node[above] {$\costs(m,m,\lambda_2)$} (5)
    (3) edge (6)
	edge (7)
    (4) edge (6)
	edge node[below] {$\costs(0,0,\lambda_2)$} (7)
    (8) edge node[above right=-0.15cm] {$0=\costs(m,0,\overbrace{\lambda_{T+1}}^{0})$} (11)
    (9) edge node[above] {$0$} (11)
    (10) edge node[above] {$0$} (11)

    (1) edge ($(1)!.2!(8)$)
    (2) edge ($(2)!.4!(6)$)
    (4) edge ($(4)!.4!(5)$)
    (3) edge ($(3)!.4!(5)$)
    ($(3)!.6!(5)$) edge (5)
    ($(2)!.6!(6)$) edge (6)
    ($(2)!.6!(7)$) edge (7)

    ($(6)!.6!(8)$) edge (8)
    ($(5)!.6!(8)$) edge (8)

    ($(5)!.6!(9)$) edge (9)
    ($(6)!.6!(9)$) edge (9)
    ($(7)!.6!(9)$) edge (9)

    ($(5)!.6!(10)$) edge (10)
    ($(6)!.6!(10)$) edge (10)
    ($(7)!.6!(10)$) edge (10)

    ($(2)!.8!(11)$) edge node[above] {$0$} (11);

\end{tikzpicture}
}
\caption{Level structured graph for a pseudo-polynomial, optimal offline algorithm}
\label{fig_graph_pseudo_poly}
\end{figure}
The cost of a path $P=(v_{x_0,T_0},v_{x_1,T_0+1},\ldots,v_{x_n,T_0+n})$ with length $n$ in our graph (where $x_t\in\fromto{0}{m}, T_0\in\fromto{0}{T}$ and $T_0+n\leq T+1$) is thus given by
\begin{equation*}
	\costs(P)\coloneqq\sum\limits_{t=1}^{n}\costs(x_{t-1},x_t,\lambda_{T_0+t})
\end{equation*}
In particular, for a path $P=(v_{0,0},v_{x_1,1},\ldots,v_{x_T,T},v_{0,T+1})$ from our start node $v_{0,0}$ to our end node $v_{0,T+1}$ we have
\begin{equation}
	\costs(P)=\costs(0,x_1,\lambda_1)+\sum\limits_{t=2}^{T}\costs(x_{t-1},x_{t},\lambda_{t})+\underbrace{\costs(x_T,0,0)}_{=0}=\costs(0,x_1,\lambda_1)+\sum\limits_{t=2}^{T}\costs(x_{t-1},x_{t},\lambda_{t})\label{eq_v0_VT1_path_costs}
\end{equation}
Note that the cost of such a path directly correspond to those of a schedule $\mx$ (see~\eqref{eq_mx_schedule_total_costs}).
Any shortest path from $v_{0,0}$ to $v_{0,T+1}$ is thus forced to minimise the cost of the corresponding schedule. Needless to say, this demands for a proof of correctness.
\begin{lem}\label{lem_sched_path_pseudo_poly}
Let $\bm{\mx}$ be the set of all schedules $\mx$ for $\inp$, and let $\bm{\mathcal{P}}$ the set of all paths from $v_{0,0}$ to $v_{0,T+1}$. The map
\begin{equation*}
	\Phi:\bm{\mathcal{P}}\rightarrow\bm{\mx},\quad (v_{0,0},v_{x_1,1},v_{x_2,2},\ldots,v_{x_T,T},v_{0,T+1})\mapsto (x_1,\ldots,x_T)
\end{equation*}
is a bijection with inverse map
\begin{equation*}
	\Psi:\bm{\mx}\rightarrow\bm{\mathcal{P}},\quad(x_1,\ldots,x_T)\mapsto (v_{0,0},v_{x_1,1},v_{x_2,2},\ldots,v_{x_T,T},v_{0,T+1})
\end{equation*}
satisfying $\costs(\mx)=\costs\bigl(\Psi(\mx)\bigr)$.
\end{lem}
\begin{proof}
It is easy to check that $\Psi\circ\Phi=id_{\bm{\mathcal{P}}}$ and $\Phi\circ\Psi=id_{\bm{\mx}}$. Thus, $\Phi$ is indeed bijective with inverse map $\Psi$. Next, let $\mx=(x_1,\ldots,x_T)$ be a schedule for $\inp$. We have
\begin{equation*}
	P\coloneqq\Psi(\mx)=(v_{0,0},v_{x_1,1},v_{x_2,2},\ldots,v_{x_T,T},v_{0,T+1})
\end{equation*}
We examine the cost of $\mx$ and conclude
\begin{equation*}
\costs(\mx)\stackrel{\eqref{eq_mx_schedule_total_costs}}{=}\sum\limits_{t=1}^{T}\costs(x_{t-1},x_{t},\lambda_t)=\underbrace{\costs(x_0,x_1,\lambda_1)}_{=\costs(0,x_1,\lambda_1)}+\sum\limits_{t=2}^{T}\costs(x_{t-1},x_{t},\lambda_t)\stackrel{\eqref{eq_v0_VT1_path_costs}}{=}\costs(P)
\end{equation*}
Which shows that $\Psi$ and, as a consequence, $\Phi$ are cost-preserving maps.
\end{proof}
We can now use this bijection to obtain our desired result: the correspondence between optimal schedules and shortest paths.
\begin{thm}\label{thm_opt_sched_short_path_pseudo_poly}
There is a cost-preserving bijection between optimal schedules $\mx$ for $\inp$ and shortest paths from $v_{0,0}$ to $v_{0,T+1}$.
\end{thm} 
\begin{proof}
By Lemma~\ref{lem_sched_path_pseudo_poly}, we have a bijection $\Psi$ between schedules $\mx$ and paths from $v_{0,0}$ to $v_{0,T+1}$ obeying $\costs(\mx)=\costs\bigl(\Psi(\mx)\bigr)$. Thus, we have 
\begin{equation*}
	\costs(\mx)\text{ minimal}\iff \costs\bigl(\Psi(\mx)\big)\text{ minimal}
\end{equation*}
and the claim follows.
\end{proof}
In the following, we give an algorithm based on our just verified construction. 
We split our procedure into two subroutines. 

\textproc{shortest\_paths} layerwise calculates the minimum costs to the nodes of our graph; that is, it calculates the shortest paths following the topological sorting of the graph. It returns the minimum costs to all nodes as well as the predecessor of any node in respect to its shortest path.

\textproc{extract\_schedule} uses the predecessors calculated by \textproc{shortest\_paths} in order to obtain the sequence of nodes describing a shortest path and thereby a schedule for our problem instance.
\begin{algorithm}[H]
  \caption{Pseudo-polynomial optimal offline scheduling}
  \begin{algorithmic}[1]
  \Function{optimal\_offline\_scheduling}{$m,T,\Lambda,\beta,f$}
	  \Let{$(C,P)$}{\Call{shortest\_paths}{$m,T,\Lambda,\beta,f$}}
	  \Let{$\mx$}{\Call{extract\_schedule}{$P,T$}}
	  \State \Return{$\mx$}
  \EndFunction
  \Statex
  \Function{shortest\_paths}{$m,T,\Lambda,\beta,f$}
	\Blet{$C[T+1,m]$ and $P[T+1,m]$}{new arrays}\Comment{Costs and predecessors of nodes}
	\For{$x \gets 0 \textrm{ to } m$}\Comment{Initialisation}
		\Let{$C[1,x]$}{$\costs(0,x,\lambda_1)$}
		\Let{$P[1,x]$}{$0$}
	\EndFor
	\For{$t \gets 2 \textrm{ to } T$}\Comment{Iterative calculation of costs and predecessors}
		\For{$x' \gets 0 \textrm{ to } m$}
			\Let{$P[t,x']$}{$\argmin\limits_{x\in\fromto{0}{m}}\bigl\{C[t-1,x]+\costs(x,x',\lambda_t)\bigr\}$}\Comment{Find best preceding choice}
			\Let{$C[t,x']$}{$C\bigl[t-1,P[t,x']\bigr]+c\bigl(P[t,x'],x',\lambda_t\bigr)$}
		\EndFor
	\EndFor
	\Let{$P[T+1,0]$}{$\argmin\limits_{x\in\fromto{0}{m}}\bigl\{C[T,x]\bigr\}$}\Comment{Find best choice for last time slot}
	\Let{$C[T+1,0]$}{$C\bigl[T,P[T+1,0]\bigr]$}
	\State \Return{$(C,P)$}
  \EndFunction
  \Statex
  \Function{extract\_schedule}{$P,T$}
	\Blet{$\mx[T]$}{a new array}
    	\Let{$\mx[T]$}{$P[T+1,0]$}\Comment{Get best choice for last time slot}
        \For{$t \gets T-1 \textrm{ to } 1$}\Comment{Iteratively obtain a schedule by using the predecessors}
		\Let{$\mx[t]$}{$P\bigl[t+1,\mx[t+1]\bigr]$}
	\EndFor
	\State \Return{$\mx$}
  \EndFunction

  \end{algorithmic}
\label{alg_opt_offline_pseudo_poly}
\end{algorithm}
The correctness of Algorithm~\ref{alg_opt_offline_pseudo_poly} directly follows from the correctness of the shortest path calculation for directed acyclic graphs (for a proof see~\parencite{intro-algo}) and from Theorem~\ref{thm_opt_sched_short_path_pseudo_poly}.

Naturally, we are interested in our algorithm's time and memory complexity.
For this, we first need to consider the size of our input $\inp=(m,T,\Lambda,\beta,f)$. In theory, our function $f:[0,1]\rightarrow \mathbb{R}$ may be, for example, easily defined by saying $f(x)\coloneqq x^2$; however, practically, it is difficult to answer the question what the size of such a function as part of the input may be. For simplicity, we consider the size of $f$ negligible in comparison with the size of the remaining input parts. Further, we assume a real number $r$ to require $\Theta\bigl(\log_2(r)\bigr)$ bits for its encoding.\unsure{Real numbers? How should I write this? What about $\log_2(0)$?} The size of our input is then given by\unsure{The following calculation. $\mathcal{O}(m)$?}
\begin{align}
	size(\inp)&=size(m)+size(T)+size(\Lambda)+size(\beta)\nonumber\\
	&=\Theta\bigl(\log_2(m)\bigr)+\Theta\bigl(\log_2(T)\bigr)+\sum\limits_{i=1}^T \Theta\bigl(\log_2(\lambda_i)\bigr)+\Theta\bigl(\log_2(\beta)\bigr)\nonumber\\
	&=\Theta\bigl(\log_2(m)+\log_2(T)+\log_2(\beta)\bigr)+\sum\limits_{i=1}^T \Theta\bigl(\log_2(\overbrace{\lambda_i}^{\le m})\bigr)\nonumber\\
	&\in\Theta\bigl(\log_2(m)+\log_2(T)+\log_2(\beta)\bigr)+\mathcal{O}\bigl(T\log_2(m)\bigr)\nonumber\\
	&\in\mathcal{O}\bigl(T\log_2(m)+\log_2(\beta)\bigr)\label{eq_inp_size}
\end{align}

Subroutine \textproc{shortest\_paths} needs $\Theta(m)$ iterations for its initialisation, $\Theta(Tm^2)$ steps for the iterative calculation, and $\Theta(m)$ steps for its final minimisation search. In addition, \textproc{extract\_schedule} needs $\Theta(T)$ iterations for its schedule retrieval. Thus, we receive a runtime of 
\begin{equation*}
	\Theta(m+Tm^2+m+T)=\Theta(Tm^2)
\end{equation*}
The runtime is polynomial in the numeric value of $m$ and $T$; however, it is exponential in the size of the input because, as we saw in~\eqref{eq_inp_size}, we only need $\log_2(m)$ bits to encode $m$. Hence, the algorithm is pseudo-polynomial.

Our memory demand is defined by the size of the arrays $C$ and $P$. As both arrays are of size $\Theta(Tm)$, we have a memory complexity of $\Theta(Tm)$\unsure{Is this enough?}

\section{A pseudo-linear algorithm}\label{sec_opt_offline_pseudo_lin}
The algorithm developed in section~\ref{sec_opt_offline_pseudo_poly} is of a quite simple nature. Its underlying graph $G$ is able to represent any possible schedule $\mx$ since we simply we add an edge for any possible scheduling choice at any possible time slot. This approach seems rather intuitive and readily verifiable, but this convenience comes with a cost; the density of $G$ causes a quadratic runtime in the number of servers $m$. In order to improve the runtime to pseudo-linear complexity, we need to thin out our graph. 

Let us revise our initial algorithm. Our graph consists of nodes $v_{x,t}$. Any node $v_{x,t}$ denotes the state of distributing the arrival rate $\lambda_{t}$ evenly to $x$ servers in time slot $t$. The algorithm calculates the minimum costs to all nodes. The cost of a node $v_{x,t}$ thus corresponds to the lowest achievable cost up to time slot $t$ of all schedules $\mx$ that assign $x$ servers at time $t$ to process the arrival rate $\lambda_t$. In particular, the cost of the end node $v_{0,T+1}$ tell us the minimum cost of all schedules. This approach, however, does not consider the possibility to schedule $y\neq x$ servers in time slot $t$ and to switch to $x$ servers just at the very last moment of $t$ when calculating the cost of $v_{x,t}$. Consider the following example:
\begin{exmpl}
Let $\inp_i=(m=1,T=2,\Lambda_i,\beta=1,f)$ be the inputs for two problem instances with $f(x)=x^2+1$, $\Lambda_1=(1,0)$, and $\Lambda_2=(0,1)$. Below we illustrate the graphs for the two problem instances and the corresponding calculations done by the algorithm derived in section~\ref{sec_opt_offline_pseudo_poly}.
\begin{figure}[H]
\captionsetup{farskip=-1em} 
\captionsetup[subfigure]{labelformat=empty}
\centering
\subfloat[\underline{Problem $\inp_1$:} State $v_{0,1}$ could be reached with cost $3$ by moving down from node $v_{1,1}$.]{
\scalebox{0.75}{
  \begin{tikzpicture}[->,>=stealth',auto,node distance=2cm,thick,node/.style={minimum size=1cm,circle,draw}]
  \node[node, label={[color=red]$0$}] (1) {0,0};
  \node[node, label={[below=1.05cm,color=red]$\infty$}] (2) [below right =of 1] {0,1};
  \node[node, label={[color=red]$3$}] (3) [above right=of 1] {1,1};
  \node[node, label={[below=1.05cm,color=red]$3$}] (4) [right =of 2] {0,2};
  \node[node, label={[color=red]$4$}] (5) [right =of 3] {1,2};
  \node[node, label={[color=red]$3$}] (6) [below right =of 5] {0,3};

  \path[every node/.style={font=\sffamily\small}]
    (1) edge node[below left] {$\infty$} (2)
	edge node[above left] {$3$} (3)
    (2) edge node[below] {$0$} (4)
    (2) edge node[label={[xshift=0.88cm, yshift=0.9cm]$2$}] {} (5)
    (3) edge node[label={[xshift=0.55cm, yshift=-1.82cm]$0$}] {} (4)
    (3) edge node[above] {$1$} (5)
    (4) edge node[below right] {$0$} (6)
    (5) edge node[above right] {$0$} (6)
    (3) edge[dashed] node[left] {$0$} (2);
\end{tikzpicture}
}}
\qquad
\subfloat[\underline{Problem $\inp_2$:} State $v_{1,1}$ could be reached with cost $1$ by moving up from $v_{0,1}$.]{
\scalebox{0.75}{
  \begin{tikzpicture}[->,>=stealth',auto,node distance=2cm,thick,node/.style={minimum size=1cm,circle,draw}]
  \node[node, label={[color=red]$0$}] (1) {0,0};
  \node[node, label={[below=1.05cm,color=red]$0$}] (2) [below right =of 1] {0,1};
  \node[node, label={[color=red]$2$}] (3) [above right=of 1] {1,1};
  \node[node, label={[below=1.05cm,color=red]$\infty$}] (4) [right =of 2] {0,2};
  \node[node, label={[color=red]$3$}] (5) [right =of 3] {1,2};
  \node[node, label={[color=red]$3$}] (6) [below right =of 5] {0,3};

  \path[every node/.style={font=\sffamily\small}]
    (1) edge node[below left] {$0$} (2)
	edge node[above left] {$2$} (3)
    (2) edge node[below] {$\infty$} (4)
    (2) edge node[label={[xshift=0.88cm, yshift=0.9cm]$3$}] {} (5)
    (3) edge node[label={[xshift=0.55cm, yshift=-1.82cm]$\infty$}] {} (4)
    (3) edge node[above] {$2$} (5)
    (4) edge node[below right] {$0$} (6)
    (5) edge node[above right] {$0$} (6)
    (2) edge[dashed] node[left] {$1=\beta$} (3);
\end{tikzpicture} }}
\caption{Two examples depicting a shortcoming of our initial approach. The calculated nodes' costs are highlighted in red. Dashed edges are not part of the initial algorithm.}
\end{figure}
\end{exmpl}
Although our algorithm delivers the correct end results, its immediate steps are somewhat unsatisfying. We want our states to capture a more general notion than given in section~\ref{sec_opt_offline_pseudo_poly}; preferably, we would like a node $v_{x,t}$ to denote the state of having $x$ active servers at the end of time slot $t$. In practice, we may reach a state $v_{x,t}$ by moving down from a state $v_{y^\downarrow,t}$ where $y^\downarrow>x$ with cost $0$ or by moving up from a state $v_{y^\uparrow,t}$ where $y^\uparrow<x$ with cost $\beta(x-y^\uparrow)$.

In order to allow for these new possibilities, given a problem instance $\inp$, we define a directed, weighted graph as follows:
\begin{align*}
	&V\coloneqq\bigl\{v_{x,t\downarrow}\mid x\in\fromto{0}{m},t\in[T]\bigr\}\dotcup\bigl\{v_{x,t\uparrow}\mid x\in\fromto{0}{m},t\in[T-1]\bigr\}\dotcup\{v_{0,0}\}\\
	&E_s\coloneqq\bigl\{(v_{0,0},v_{x,1\downarrow})\mid x\in\fromto{0}{m}\bigr\}\\
	&E_\downarrow\coloneqq\bigl\{(v_{x,t\downarrow},v_{x-1,t\downarrow})\mid x\in[m],t\in[T]\bigr\}\\
	&E_\uparrow\coloneqq\bigl\{(v_{x-1,t\uparrow},v_{x,t\uparrow})\mid x\in[m],t\in[T-1]\bigr\}\\
	&E_{\downarrow\uparrow}\coloneqq\bigl\{(v_{x,t\downarrow},v_{x,t\uparrow})\mid x\in\fromto{0}{m},t\in[T-1]\bigr\}\\
	&E_{\uparrow\downarrow}\coloneqq\bigl\{(v_{x,t\uparrow},v_{x,t+1\downarrow})\mid x\in\fromto{0}{m},t\in[T-1]\bigr\}\\
	&E\coloneqq E_s\dotcup E_\downarrow\dotcup E_\uparrow\dotcup E_{\downarrow\uparrow}\dotcup E_{\uparrow\downarrow}\\
	&c_G(e)\coloneqq
	\begin{cases}
		\costs(0,x,\lambda_1), & \text{if $e=(v_{0,0},v_{x,1\downarrow})\in E_s$}\\
		\opcosts(x,\lambda_{t+1}), & \text{if $e=(v_{x,t\uparrow},v_{x,t+1\downarrow})\in E_{\uparrow\downarrow}$}\\
		\beta, & \text{if $e\in E_\uparrow$}\\
		0, & \text{if $e\in(E_\downarrow\dotcup E_{\downarrow\uparrow})$}
	\end{cases}\\
	&G\coloneqq(V,E,c_G)
\end{align*}
%\begin{textblock}{3}(10,4.5)
%\begin{framed}
A more appealing, graphical representation can be found in the following figure.
%As the phrase goes, \textit{``A picture is worth a thousand words.''} See Figure~\ref{fig_graph_pseudo_lin} for a more appealing illustration.%TODO set position!!
%\end{framed}
%\end{textblock}
\begin{figure}[H]
\centering
\scalebox{0.6}{
\begin{tikzpicture}[->,>=stealth',auto,node distance=2.2cm,thick,node/.style={minimum size=1.5cm,circle,draw}]
  \node[node] (1) {0,0};
  \node[node] (4) [below right=4cm of 1] {0,1$\downarrow$};
  \node[node] (3) [above =0.5cm of 4] {1,1$\downarrow$};
  \node[node] (2) [above right=4cm of 1] {m,1$\downarrow$};
  \node[node] (5) [below =0.5cm of 2] {m-1,1$\downarrow$};
  \node[node] (7) [right =of 3] {1,1$\uparrow$};
  \node[node] (6) [right =of 2] {m,1$\uparrow$};
  \node[node] (8) [right =of 4] {0,1$\uparrow$};
  \node[node] (9) [right =of 5] {m-1,1$\uparrow$};
  \node[node] (11) [right =of 7] {1,2$\downarrow$};
  \node[node] (10) [right =of 6] {m,2$\downarrow$};
  \node[node] (12) [right =of 8] {0,2$\downarrow$};
  \node[node] (13) [right =of 9] {m-1,2$\downarrow$};
  \node[node] (15) [right =of 11] {1,2$\uparrow$};
  \node[node] (14) [right =of 10] {m,2$\uparrow$};
  \node[node] (16) [right =of 12] {0,2$\uparrow$};
  \node[node] (17) [right =of 13] {m-1,2$\uparrow$};
  \node[node] (19) [right =5cm of 15] {1,T$\downarrow$};
  \node[node] (18) [right =5cm of 14] {m,T$\downarrow$};
  \node[node] (20) [right =5cm of 16] {0,T$\downarrow$};
  \node[node] (21) [right =5cm of 17] {m-1,T$\downarrow$};

  \node (22) at ($(2)!.5!(4)$) {};
  \node (23) at ($(2)!.487!(4)$) {\vdots};
  \node (24) at ($(6)!.487!(8)$) {\vdots};
  \node (25) at ($(10)!.487!(12)$) {\vdots};
  \node (26) at ($(14)!.487!(16)$) {\vdots};
  \node at ($(18)!.487!(20)$) {\vdots};
  \node at ($(23)!.487!(24)$) {\vdots};
  \node at ($(24)!.487!(25)$) {\vdots};
  \node at ($(25)!.487!(26)$) {\vdots};

  \node (27) at ($(14)!.5!(18)$) {\ldots};
  \node at ($(15)!.5!(19)$) {\ldots};
  \node (28) at ($(16)!.5!(20)$) {\ldots};
  \node at ($(17)!.5!(21)$) {\ldots};
  \node at ($(27)!.5!(28)$) {\ldots};

  \path[every node/.style={font=\sffamily\small}]
    (1) edge[red] node[black,above left] {$\costs(0,m,\lambda_1)$} (2)
	edge node[label={[xshift=0.8cm, yshift=-0.9cm]$\costs(0,m-1,\lambda_1)$}] {} (5)
	edge node[above right=-0.13cm] {$\costs(0,1,\lambda_1)$} (3)
	edge node[below left] {$\costs(0,0,\lambda_1)$} (4)
    (1) edge ($(1)!.82!(22)$)

    (2) edge node[above] {$0$} (6)
    (3) edge node[above] {$0$} (7)
    (4) edge[red] node[black,above] {$0$} (8)
    (5) edge node[above] {$0$} (9)

    (2) edge[red] node[black,right] {$0$} (5)
    (3) edge[red] node[black,right] {$0$} (4)

    (9) edge[red] node[black,right] {$\beta$} (6)
    (8) edge[red] node[black,right] {$\beta$} (7)

    (6) edge[red] node[black,above] {$\opcosts(m,\lambda_2)$} (10)
    (7) edge node[above] {$\opcosts(1,\lambda_2)$} (11)
    (8) edge node[above] {$\opcosts(0,\lambda_2)$} (12)
    (9) edge node[above] {$\opcosts(m-1,\lambda_2)$} (13)

    (10) edge node[above] {$0$} (14)
    (11) edge node[above] {$0$} (15)
    (12) edge[red] node[black,above] {$0$} (16)
    (13) edge node[above] {$0$} (17)

    (10) edge[red] node[black,right] {$0$} (13)
    (11) edge[red] node[black,right] {$0$} (12)

    (17) edge[red] node[black,right] {$\beta$} (14)
    (16) edge[red] node[black,right] {$\beta$} (15)

    (5) edge[red] node[black,right] {$0$} ($(5)!.35!(3)$)
    ($(5)!.65!(3)$) edge[red] node[black,right] {$0$} (3)

    ($(7)!.65!(9)$) edge[red] node[black,right] {$\beta$} (9)
    (7) edge[red] node[black,right] {$\beta$} ($(7)!.35!(9)$)
 
    (13) edge[red] node[black,right] {$0$} ($(13)!.35!(11)$)
    ($(13)!.65!(11)$) edge[red] node[black,right] {$0$} (11)

    ($(15)!.65!(17)$) edge[red] node[black,right] {$\beta$} (17)
    (15) edge[red] node[black,right] {$\beta$} ($(15)!.35!(17)$)


    (14) edge[red] node[label={[black,xshift=0cm, yshift=-0.26cm]$\opcosts(m,\lambda_3)$}] {} ($(14)!.4!(18)$)
    (15) edge node[label={[xshift=0cm, yshift=-0.26cm]$\opcosts(1,\lambda_3)$}] {} ($(15)!.4!(19)$)
    (16) edge node[label={[xshift=0cm, yshift=-0.26cm]$\opcosts(0,\lambda_3)$}] {} ($(16)!.4!(20)$)
    (17) edge node[label={[xshift=0.3cm, yshift=-0.26cm]$\opcosts(m-1,\lambda_3)$}] {} ($(17)!.4!(21)$)

    ($(14)!.6!(18)$) edge[red] node[black,above] {$\opcosts(m,\lambda_T)$} (18)
    ($(15)!.6!(19)$) edge node[above] {$\opcosts(1,\lambda_T)$} (19)
    ($(16)!.6!(20)$) edge node[above] {$\opcosts(0,\lambda_T)$} (20)
    ($(17)!.6!(21)$) edge node[label={[xshift=-0.2cm, yshift=-0.26cm]$\opcosts(m-1,\lambda_T)$}] {} (21)

    (18) edge[red] node[black,right] {$0$} (21)
    (19) edge[red] node[black,right] {$0$} (20)
    (21) edge[red] node[black,right] {$0$} ($(21)!.35!(19)$)
    ($(21)!.65!(19)$) edge[red] node[black,right] {$0$} (19);
\end{tikzpicture}
}
\caption{Graph for a pseudo-linear, optimal offline algorithm; the path of the topological sorting is highlighted in red.}
\label{fig_graph_pseudo_lin}
\end{figure}
For any possible amount of active servers $x$ and any time slot $t$, we add a node $v_{x,t\downarrow}$. Semantically, the cost of $v_{x,t\downarrow}$ will denote the minimum cost up to time slot $t$ when processing $\lambda_t$ with $x$ or more servers. Further, for any time slot $t\in[T-1]$ we add a node $v_{x,t\uparrow}$. The cost of $v_{x,t\uparrow}$ will denote the minimum cost of having $x$ active servers at the end of time slot $t$. Moreover, we add a start node $v_{0,0}$. Our end node will be $v_{0,T\downarrow}$.

The set of edges $E_s$ denotes the start initialisation step. An edge $(v_{x,t\uparrow},v_{x,t+1\downarrow})\in E_{\uparrow\downarrow}$ accounts for the operating costs that incur when processing the arrival rate $\lambda_{t+1}$ with $x$ active servers.

After any time step from $t-1$ to $t$ that incurs operating costs, we have a minimisation step in our graph. For this, we first move down the chain $v_{m,t\downarrow},v_{m-1,t\downarrow},\ldots,v_{0,t\downarrow}$ using edges from $E_\downarrow$ with cost $0$. Then we proceed to move to the right from $v_{0,t\downarrow}$ to $v_{0,t\uparrow}$. Lastly, we move up the chain $v_{0,t\uparrow},v_{1,t\uparrow},\ldots,v_{m,t\uparrow}$ using edges from $E_\uparrow$ with cost $\beta$. In order to have the possibility to keep the calculated costs of $v_{x,t\downarrow}$ during the upward movement, we add edges $(v_{x,t\downarrow},v_{x,t\uparrow})\in E_{\downarrow\uparrow}$ with cost $0$.

This minimisation step is the key to our runtime improvement. It facilitates the determination of the best predecessor of a state $v_{x,t\downarrow}$ because we already know that the minimum cost of having $x$ servers at the end of time slot $t-1$ is stored in $v_{x,t-1\uparrow}$. Thus, the cheapest possibility to process the next arrival rate $\lambda_t$ using $x$ servers can simply be calculated by adding $\opcosts(x,\lambda_t)$ to the cost of $v_{x,t-1\uparrow}$. Consequently, the cost of $v_{x,t\downarrow}$ is given by the minimum of $v_{x,t-1\uparrow}+\opcosts(x,\lambda_t)$ and $v_{x+1,\downarrow}$.

As one can see in Figure~\ref{fig_graph_pseudo_lin}, we stretched our graph but at the same time also greatly reduced the amount of edges compared to our initial approach in section~\ref{sec_opt_offline_pseudo_poly}. By following the coloured path of the topological sorting, we can work our way through the graph to calculate the shortest paths, ultimately reaching the destination $v_{0,T\downarrow}$. The cost of our destination $v_{0,T\downarrow}$ will denote the minimum cost up to time slot $T$ when processing $\lambda_T$ with 0 or more servers. Hence, it will contain our desired end result --- the minimum cost of all possible schedules.

Our next task shall be the verification of our new construction. For this, we first examine the possible paths from $v_{0,0}$ to $v_{0,T\downarrow}$ in our graph.
In contrast to our approach in section~\ref{sec_opt_offline_pseudo_poly}, our new graph contains paths that do not directly correspond to a schedule $\mx$. For example, consider the following path:
\begin{equation*}
	P\coloneqq(v_{0,0},v_{1,1\downarrow},v_{0,1\downarrow},v_{0,1\uparrow},v_{1,1\uparrow},v_{2,1\uparrow},v_{2,2\downarrow},\ldots,v_{0,T\downarrow})
\end{equation*}
A schedule corresponding to $P$ would use one active server in its first time slot, then power down this server, and subsequently turn on two servers to process the next arrival rate. This seems somewhat unreasonable. We could just keep the initial server on to save switching costs (note the correspondence to Proposition~\ref{prop_reasonable_switching}). In fact, this behaviour cannot even be represented using our schedule notation $\mx$ and optimisation condition~\eqref{eq_mx_schedule_total_costs}. We can, however, modify $P$ to represent a more reasonable sequence, that is we set
\begin{equation*}
	P'\coloneqq(v_{0,0},v_{1,1\downarrow},v_{1,1\uparrow},v_{2,1\uparrow},v_{2,2\downarrow},\ldots,v_{0,T\downarrow})
\end{equation*}
This revised path now pleasantly translates to a schedule $\mx$, in this case $\mx=(1,2,\ldots)$. We now want to give a more formal definition of our observation.
\begin{defn}[Reasonable paths]\label{defn_reasn_paths}
Let $P$ be a path from $v_{0,0}$ to $v_{0,T\downarrow}$. For any time slot $t\in[T]$, let $E_\downarrow^t$ be the set of edges $(v_{x,t\downarrow},v_{x-1,t\downarrow})\in E_\downarrow$ used by $P$ at time $t$. Similarly, let $E_\uparrow^t$ be the set of edges $(v_{x,t\uparrow},v_{x+1,t\uparrow})\in E_\uparrow$ used by $P$ at time $t\in[T-1]$.
The path $P$ is called \textit{reasonable} if for any time slot $t\in[T-1]$, the path does not shut down and power on servers simultaneously at $t$. More formally, $P$ must satisfy the formula
\begin{equation}
	\forall t\in[T-1]\left(E_\downarrow^t=\emptyset \lor E_\uparrow^t=\emptyset\right)\label{eq_reasn_path}
\end{equation}
\end{defn}
Using this definition, we can now verify that our reasonable paths indeed deserve to be called \textit{reasonable}.
\begin{prop}\label{prop_path_to_reasn_path}
Any given path $P$ from $v_{0,0}$ to $v_{0,T\downarrow}$ can be transformed to a reasonable path $P'$ with $\costs(P')\le\costs(P)$.
\end{prop}
\begin{proof}
Let $P$ be a path from $v_{0,0}$ to $v_{0,T\downarrow}$. We give a procedure that repeatedly modifies $P$ such that it satisfies~\eqref{eq_reasn_path} and reduces or retains its costs. 

Let $t\in[T-1]$ be the first time slot falsifying~\eqref{eq_reasn_path}. If there does not exist such a time slot, we are done. Otherwise, let $E_\downarrow^t$ and $E_\uparrow^t$ be its sets of edges as defined in Definition~\ref{defn_reasn_paths}. Since $P$ falsifies~\eqref{eq_reasn_path} at time $t$, both $E_\downarrow^t$ as well as $E_\uparrow^t$ must be non-empty. Thus, we can obtain the ``maximum'' nodes involved in these sets.
\begin{align*}
	x_s\coloneqq&\max\bigl\{x\in[m]\mid (v_{x,t\downarrow},v_{x-1,t\downarrow})\in E_\downarrow^t\bigr\}\\
	x_e\coloneqq&\max\bigl\{x\in[m]\mid (v_{x-1,t\uparrow},v_{x,t\uparrow})\in E_\uparrow^t\bigr\}
\end{align*}
Next, consider the subpath $S=(v_{x_s,t\downarrow},\ldots,v_{x_e,t\uparrow})$ of $P$. Note that the subpath in particular uses all edges from $E_\downarrow^t$ and $E_\uparrow^t$.
Naturally, a most cost-efficient path $S'$ from node $v_{x_s,t\downarrow}$ to $v_{x_e,t\uparrow}$ minimises the number of edges $(v_{x,t\uparrow},v_{x+1,t\uparrow})$ since each of these edges incurs cost $\beta\ge 0$. For this, we observe that we must not shut down servers if $x_s\le x_e$, and that we must not power on servers if $x_s\ge x_e$; thus, we consider three cases for $S'$:
\begin{equation*}
	S'\coloneqq
	\begin{cases}
		(v_{x_s,t\downarrow},v_{x_s,t\uparrow},v_{x_s+1,t\uparrow},\ldots,v_{x_e,t\uparrow}), & \text{if $x_s< x_e$}\\
		(v_{x_s,t\downarrow},v_{x_s-1,t\downarrow},\ldots,v_{x_e,t\downarrow},v_{x_e,t\uparrow}), & \text{if $x_s>x_e$}\\
		(v_{x_s,t\downarrow},v_{x_e,t\uparrow}), & \text{if $x_s=x_e$}
	\end{cases}
\end{equation*}
In each case, $S'$ uses edges from at most one of the sets $E_\downarrow^t$ and $E_\uparrow^t$. Thus, by replacing the subpath $S$ of $P$ with $S'$, we obtain a new schedule $P'$ that satisfies~\eqref{eq_reasn_path} up to and including time slot $t$. Moreover, the paths $P$ and $P'$ coincide in their costs before visiting $v_{x_s,t\downarrow}$ and after visiting $v_{x_e,t\uparrow}$; however, they differ in that there are less switching costs $\beta$ at time $t$ using $P'$. As we assume $\beta\ge0$, we conclude $\costs(P')\le \costs(P)$.

Hence, by repeating described process on $P'$, we obtain a terminating procedure that returns a path satisfying the conditions.
\end{proof}
As a result of Proposition~\ref{prop_path_to_reasn_path}, we shall subsequently focus our attention on reasonable paths. Our goal is to establish the connection between reasonable paths $P$ of our graph and schedules $\mx$. Intuitively, one can see that such a path $P$ is uniquely determined by the ``maximum'' nodes $v_{x,t\downarrow}$ taken by $P$ at any time slot $t$; these nodes represent the state of processing $\lambda_t$ with $x$ servers. Consider the following example:
\begin{exmpl}
Let $\inp=\bigl(m=3,T=3,\Lambda=(3,0,1),\beta,f\bigr)$ be the input for a problem instance. Then one example of a reasonable path is given by
\begin{align*}
	P&=(v_{0,0},v_{3,1\downarrow},v_{2,1\downarrow},v_{1,1\downarrow},v_{0,1\downarrow},v_{0,1\uparrow},v_{0,2\downarrow},v_{0,2\uparrow},v_{1,2\uparrow},v_{1,3\downarrow},v_{0,3\downarrow})
\end{align*}
A schedule corresponding to $P$ would use three active server in its first time slot, then power down all servers for the second time slot, and ultimately power on one server to process the last arrival rate; that is, the corresponding schedule of $P$ is $\mx=(3,0,1)$. This sequence also corresponds to the sequence of ``maximum'' nodes $v_{x,t\downarrow}$ taken by $P$, as can be seen in the following figure.
\begin{figure}[H]
\centering
\scalebox{0.79}{
\begin{tikzpicture}[->,>=stealth',auto,node distance=2cm,thick,node/.style={minimum size=1.2cm,circle,draw}]
  \node[node] (1) {0,0};
  \node[node] (4) [below right=3.5cm of 1] {0,1$\downarrow$};
  \node[node] (3) [above =1cm of 4] {1,1$\downarrow$};
  \node[node,draw=blue] (2) [above right=3.5cm of 1] {3,1$\downarrow$};
  \node[node] (5) [below =1cm of 2] {2,1$\downarrow$};
  \node[node] (7) [right =of 3] {1,1$\uparrow$};
  \node[node] (6) [right =of 2] {3,1$\uparrow$};
  \node[node] (8) [right =of 4] {0,1$\uparrow$};
  \node[node] (9) [right =of 5] {2,1$\uparrow$};
  \node[node] (11) [right =of 7] {1,2$\downarrow$};
  \node[node] (10) [right =of 6] {3,2$\downarrow$};
  \node[node,draw=blue] (12) [right =of 8] {0,2$\downarrow$};
  \node[node] (13) [right =of 9] {2,2$\downarrow$};
  \node[node] (15) [right =of 11] {1,2$\uparrow$};
  \node[node] (14) [right =of 10] {3,2$\uparrow$};
  \node[node] (16) [right =of 12] {0,2$\uparrow$};
  \node[node] (17) [right =of 13] {2,2$\uparrow$};
  \node[node,draw=blue] (19) [right =of 15] {1,3$\downarrow$};
  \node[node] (18) [right =of 14] {3,3$\downarrow$};
  \node[node] (20) [right =of 16] {0,3$\downarrow$};
  \node[node] (21) [right =of 17] {2,3$\downarrow$};

  \path[every node/.style={font=\sffamily\small}]
    (1) edge[red] node[black, above left] {$\costs(0,3,3)$} (2)
	edge node[label={[xshift=0.4cm, yshift=-0.1cm]$\costs(0,2,3)$}] {} (5)
	edge node[label={[xshift=0.14cm, yshift=-0.25cm]$\costs(0,1,3)$}] {} (3)
	edge node[below left] {$\costs(0,0,3)$} (4)

    (2) edge node[above] {$0$} (6)
    (3) edge node[above] {$0$} (7)
    (4) edge[red] node[black,above] {$0$} (8)
    (5) edge node[above] {$0$} (9)

    (2) edge[red] node[black,right] {$0$} (5)
    (3) edge[red] node[black,right] {$0$} (4)
    (5) edge[red] node[black,right] {$0$} (3)

    (9) edge node[right] {$\beta$} (6)
    (8) edge node[right] {$\beta$} (7)
    (7) edge node[right] {$\beta$} (9)

    (6) edge node[above] {$\opcosts(3,1)$} (10)
    (7) edge node[above] {$\opcosts(1,1)$} (11)
    (8) edge[red] node[black,above] {$\opcosts(0,1)$} (12)
    (9) edge node[above] {$\opcosts(2,1)$} (13)

    (10) edge node[above] {$0$} (14)
    (11) edge node[above] {$0$} (15)
    (12) edge[red] node[black,above] {$0$} (16)
    (13) edge node[above] {$0$} (17)

    (10) edge node[right] {$0$} (13)
    (11) edge node[right] {$0$} (12)
    (13) edge node[right] {$0$} (11)

    (17) edge node[right] {$\beta$} (14)
    (16) edge[red] node[black,right] {$\beta$} (15)
    (15) edge node[right] {$\beta$} (17)

    (14) edge node[above] {$\opcosts(3,2)$} (18)
    (15) edge[red] node[black,above] {$\opcosts(1,2)$} (19)
    (16) edge node[above] {$\opcosts(0,2)$} (20)
    (17) edge node[above] {$\opcosts(2,2)$} (21)

    (18) edge node[right] {$0$} (21)
    (19) edge[red] node[black,right] {$0$} (20)
    (21) edge node[right] {$0$} (19);
\end{tikzpicture}}
\caption{Illustration of the path $P$. The path is highlighted in red. The maximum nodes $v_{x,t\downarrow}$ taken by $P$ are highlighted in blue.}
\end{figure}
Conversely, given the schedule $\mx=(3,0,1)$, is can easily be seen that there exists only one reasonable path corresponding to $\mx$, namely $P$.
\end{exmpl}
Before we approach our next lemma, which formalises these ideas, we need to make a precise definition of ``maximum'' nodes $v_{x,t\downarrow}$ used by a reasonable path.
\begin{defn}
Let $P$ be a reasonable path. For any $t\in[T]$, let $V_\downarrow^t$ be the set of nodes $v_{x,t\downarrow}$ visited by $P$. The ``maximum'' node $v_{x,t\downarrow}$ at each time slot $t\in[T]$ can then be identified by
\begin{equation*}
	x_t\coloneqq\max\bigl\{x\in\fromto{0}{m}\mid v_{x,t\downarrow}\in V_\downarrow^t\bigr\}
\end{equation*}
\end{defn}
Remember that a schedule $\mx=(x_1,\ldots,x_T)$ also uses the notation $x_t$ to identify the number of active severs at time $t$. Needless to say, this clash of names is intentional and justified by the next lemma.
\begin{lem}\label{lem_sched_path_pseudo_lin}
Let $\bm{\mx}$ be the set of all schedules $\mx$ for $\inp$, and let $\bm{\mathcal{P}}$ the set of all reasonable paths. The map
\begin{equation*}\label{prop_reasn_paths_identify}
	\Phi:\bm{\mathcal{P}}\rightarrow\bm{\mx},\quad P\mapsto (x_1,\ldots,x_T)
\end{equation*}
is a bijection satisfying $\costs(P)=\costs\bigl(\Phi(P)\bigr)$.
\end{lem}
\begin{proof}
First, we check that $\Phi$ is bijective, i.e.\ $\Phi$ is injective and surjective. 

Let $P$ and $P'$ be reasonable paths with $\Phi(P)=\Phi(P')$. Then $P$ as well as $P'$ must start with the edge $(v_{0,0},v_{x_1,1\downarrow})$. As $P$ and $P'$ are reasonable, they both satisfy $E_\downarrow^t=\emptyset \lor E_\uparrow^t=\emptyset$ for each time slot $t$. Due to this restriction, their paths between $v_{x_t,t\downarrow}$ and $v_{x_{t+1},t+1\downarrow}$ must coincide for $t\in[T-1]$. Further, the path from $v_{x_T,T}$ to $v_{0,T}$ is unique in our graph. Thus, we have $P=P'$, which shows the injectivity of $\Phi$.

Next, let $(x_1,\ldots,x_T)\in\bm{\mx}$ be a schedule for $\inp$. For any $t\in[T-1]$, we set
\begin{equation*}
	S_t\coloneqq
	\begin{cases}
		(v_{x_t,t\downarrow},v_{x_t,t\uparrow},v_{x_t+1,t\uparrow},\ldots,v_{x_{t+1},t\uparrow}), & \text{if $x_t<x_{t+1}$}\\
		(v_{x_t,t\downarrow},v_{x_t-1,t\downarrow},\ldots,v_{x_{t+1},t\downarrow},v_{x_{t+1},t\uparrow}), & \text{if $x_t>x_{t+1}$}\\
		(v_{x_t,t\downarrow},v_{x_{t+1},t\uparrow}), & \text{if $x_t=x_{t+1}$}
	\end{cases}
\end{equation*}
We then concatenate these subpaths to construct a reasonable path
\begin{equation*}
	P\coloneqq(v_{0,0},S_1,S_2,\ldots,S_{T-1},v_{x_T,T\downarrow},v_{x_T-1,T\downarrow},\ldots,v_{0,T\downarrow})
\end{equation*}
Evidently, the constructed path satisfies $\Phi(P)=\mx$, which shows the surjectivity of $\Phi$.

It remains to show that $\Phi$ is cost-preserving. For this, let $P$ be a reasonable path and let $\mx\coloneqq\Phi(P)=(x_1,\ldots,x_T)$ be its image. 
As $P$ is reasonable, we have $E_\downarrow^t=\emptyset \lor E_\uparrow^t=\emptyset$ for any time slot $t$. If $E_\uparrow^t$ is empty, we have $x_t\le x_{t-1}$. The cost between the nodes $v_{x_{t-1},t-1\downarrow}$ and $v_{x_t,t\downarrow}$ is then given by $\opcosts(x_t,\lambda_t)$. If $E_\uparrow^t$ is non-empty, we have $x_t>x_{t-1}$. In this case, the cost is given by $\beta(x_t-x_{t-1})+\opcosts(x_t,\lambda_t)$. Using this observation, the cost of $P$ can be calculated by 
\begin{align*}
	\costs(P)&=c(0,x_1,\lambda_1)+\sum\limits_{t=2}^{T}\bigl(\overbrace{\beta\max\{0,x_t-x_{t-1}\}}^{\swcosts(x_{t-1},x_t)}+\opcosts(x_{t},\lambda_t)\bigr)\\
	&\stackrel{\eqref{eq_mx_schedule_sw_costs}}{=}c(0,x_1,\lambda_1)+\sum\limits_{t=2}^{T}\bigl(\overbrace{\swcosts(x_{t-1},x_t)+\opcosts(x_{t},\lambda_t)}^{\costs(x_{t-1},x_t,\lambda_t)}\bigr)\\
	&\stackrel{\eqref{eq_mx_schedule_costs}}{=}c(0,x_1,\lambda_1)+\sum\limits_{t=2}^{T}\costs(x_{t-1},x_t,\lambda_t)=\sum\limits_{t=1}^{T}\costs(x_{t-1},x_t,\lambda_t)\stackrel{\eqref{eq_mx_schedule_total_costs}}{=}c(\mx)
\end{align*}
Which shows that $\Phi$ is a cost-preserving map.
\end{proof}
Again, we can use this bijection to obtain our desired result: the correspondence between optimal schedules and shortest, reasonable paths.
\begin{thm}\label{thm_opt_sched_reasn_path}
There is a cost-preserving bijection between optimal schedules $\mx$ for $\inp$ and shortest, reasonable paths.
\end{thm} 
\begin{proof}
By Lemma~\ref{lem_sched_path_pseudo_lin}, we have a bijection $\Phi$ between reasonable paths $P$ and schedules $\mx$ obeying $\costs(P)=\costs\bigl(\Phi(P)\bigr)$. Thus, we have 
\begin{equation*}
	\costs(P)\text{ minimal}\iff \costs\bigl(\Phi(P)\big)\text{ minimal}
\end{equation*}
and the claim follows.
\end{proof}
Naturally, common shortest path algorithms do not have any knowledge about our ``resonable'' paths. What if we obtain a shortest path that coincidentally is not reasonable? This shall not turn out to be a problem, as the next corollary shows us.
\begin{cor}
Any shortest path $P$ from $v_{0,0}$ to $v_{0,T\downarrow}$ can be transformed to an optimal schedule $\mx$ for $\inp$ with $\costs(\mx)=\costs(P)$.
\end{cor}
\begin{proof}
Let $P$ be a shortest path from $v_{0,0}$ to $v_{0,T\downarrow}$. By Proposition~\ref{prop_path_to_reasn_path}, the path $P$ can be transformed to a reasonable path $P'$ with $\costs(P')=\costs(P)$.
In turn, $P'$ corresponds to an optimal schedule $\mx$ with $\costs(\mx)=\costs(P')$ by Theorem~\ref{thm_opt_sched_reasn_path}. Thus, we have $c(\mx)=c(P)$, and the claim follows.
\end{proof}
In the following, we give an algorithm based on our just verified constructions. 
We split our procedure into two subroutines. 

\textproc{shortest\_paths} uses an approach similar to the well-known Bellman-Ford algorithm. It returns the minimum costs to all nodes as well as the predecessor of any node in respect to its shortest path.

\textproc{extract\_schedule} uses the predecessors calculated by \textproc{shortest\_paths} in order to obtain the sequence of nodes describing a shortest path and thereby a schedule for our problem instance.
\begin{algorithm}[H]
  \caption{Pseudo-linear optimal offline scheduling}
  \begin{algorithmic}[1]
  \Function{optimal\_offline\_scheduling}{$m,T,\Lambda,\beta,f$}
	  \Let{$(C,P)$}{\Call{shortest\_paths}{$m,T,\Lambda,\beta,f$}}
	  \Let{$\mx$}{\Call{extract\_schedule}{$P,T$}}
	  \State \Return{$\mx$}
  \EndFunction
  \Statex
  \Function{shortest\_paths}{$m,T,\Lambda,\beta,f$}
	\Blet{$C[T+1,m]$ and $P[T+1,m]$}{new arrays}\Comment{Costs and predecessors of nodes}
	\For{$x \gets 0 \textrm{ to } m$}\Comment{Initialisation}
		\Let{$C[1,x]$}{$\costs(0,x,\lambda_1)$}
		\Let{$P[1,x]$}{$0$}
	\EndFor
	\For{$t \gets 2 \textrm{ to } T$}\Comment{Iterative calculation of costs and predecessors}
		\For{$x' \gets 0 \textrm{ to } m$}
			\Let{$P[t,x']$}{$\argmin\limits_{x\in\fromto{0}{m}}\bigl\{C[t-1,x]+\costs(x,x',\lambda_t)\bigr\}$}\Comment{Find best preceding choice}
			\Let{$C[t,x']$}{$C\bigl[t-1,P[t,x']\bigr]+c\bigl(P[t,x'],x',\lambda_t\bigr)$}
		\EndFor
	\EndFor
	\Let{$P[T+1,0]$}{$\argmin\limits_{x\in\fromto{0}{m}}\bigl\{C[T,x]\bigr\}$}\Comment{Find best choice for last time slot}
	\Let{$C[T+1,0]$}{$C\bigl[T,P[T+1,0]\bigr]$}
	\State \Return{$(C,P)$}
  \EndFunction
  \Statex
  \Function{extract\_schedule}{$P,T$}
	\Blet{$\mx[T]$}{a new array}
    	\Let{$\mx[T]$}{$P[T+1,0]$}\Comment{Get best choice for last time slot}
        \For{$t \gets T-1 \textrm{ to } 1$}\Comment{Iteratively obtain a schedule by using the predecessors}
		\Let{$\mx[t]$}{$P\bigl[t+1,\mx[t+1]\bigr]$}
	\EndFor
	\State \Return{$\mx$}
  \EndFunction
  \end{algorithmic}
\label{alg_opt_offline_pseudo_linear}
\end{algorithm}
