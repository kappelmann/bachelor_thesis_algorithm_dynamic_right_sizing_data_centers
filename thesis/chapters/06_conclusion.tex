% !TeX root = ../main.tex
% Add the above to each chapter to make compiling the PDF easier in some editors.

\chapter{Conclusion}\label{chap:conclusion}

\section{Summary}
Our work started with a fairly general data center model consisting of a homogeneous collection of servers with fixed switching costs and convex operating costs. The data center's load varies with time. Consequently, the number of active servers and the workload distribution have to be dynamically adapted to minimize the data center's incurring costs. To accomplish this task, we first transferred our model's optimization problem to a simpler form in Chapter~\ref{chap:preliminaries} by exploiting discovered patterns and properties of optimal schedules. 

We then observed in Chapter~\ref{chap:opt_offline_schedule} that the data center's optimization problem can be reduced to the shortest path problem of an acyclic graph. This observation was then used to derive an optimal algorithm with pseudo-polynomial time complexity $\Theta(Tm^2)$, which in a second step was improved to pseudo-linear time $\Theta(Tm)$ by adapting the graph's structure.

Lastly, we sought to reduce the algorithm's time complexity to polynomial time. For this, we had to move to approximative methods and slightly adapt our assumptions concerning the servers' operating costs function $f$ in Chapter~\ref{chap:approx_offline_scheduling}. More specifically, we assumed $f$ to be non-negative and monotonically increasing; these assumptions, however, do not interfere with practical applicability, as discussed in Section~\ref{sec:approx_2_opt}. We first derived a 2-optimal algorithm with linear time complexity $\Theta\bigl(T\log_2(m)\bigr)$ by modifying the previously used graph. We then proved that this approach can be generalized to a $(1+\beps)$-optimal algorithm with linear time complexity $\Theta\left(T\,\frac{\log(m)}{\log(1+\beps)}\right)$, able to approximate optimal solutions with arbitrary precision.

It is noteworthy that all our derived algorithms provide integer solutions and can be easily implemented as a computer algorithm. In fact, a pseudo-code implementation for each developed and verified algorithm is given at the end of each respective section.

\section{Future Work}
Throughout this thesis, we assumed the data centers' servers to be homogeneous. As a consequence, we were able to focus on the sheer number of active servers and not on the exact set of servers which are used for the scheduling process. Further, we only considered servers with two different power states (active or shut-down). Modern machines, however, contain more than these two states, that is they contain multiple sleep states with varying energy consumption. It is thus an interesting question whether our approach can be generalized to models with more than one homogeneous server collection or multiple sleep states. 
Moreover, since all our algorithms deal with the offline variant of the data centers' scheduling problem, it is natural to ask whether our approaches can in some way be modified to work as an online algorithm. Lastly, since we did not discover a polynomial-time optimal algorithm, it remains an open question whether there exists a polynomial-time solution for our problem setting or it is an $\textbf{NP}$ problem.

One possible application of our procedures may be the online algorithm developed by Lin~et~al.~\parencite{dyn-right-sizing}. Their procedure uses a forward recurrence relation for which progressively larger convex programs have to be solved at each time step. At some time $t$, these convex programs calculate the optimal offline solution up to time $t$ for two different models: one which charges its switching costs $\beta$ when powering up servers, and one which charges~$\beta$ when shutting down servers. While the former case can be simply calculated by our derived optimal offline algorithm, the latter merely requires a small adaption of the algorithm~(see also Section~\ref{sec:model_descr}). These convex programs can thus be exchanged by two instances of our optimal offline algorithm if -- and this has not been proven so far -- Lin's approach and proofs still work with integer solutions. If this is indeed the case, the combination of Lin's online algorithm and our approximative offline solution may be another topic worth to examine.
