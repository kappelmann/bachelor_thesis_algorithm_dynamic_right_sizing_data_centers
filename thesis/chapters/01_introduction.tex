% !TeX root = ../main.tex
% Add the above to each chapter to make compiling the PDF easier in some editors.

\chapter{Introduction}
TODO: Hardware prices vs.\ energy costs in data centers, related work and purpose of this paper (offline algorithm, approximation algorithm,\ldots).
TODO: definition of optimal and x-approximation algorithm

\section{Model Description}\label{sec:model_descr}
In order to address the described rise of energy consumption, we want to examine a scheduling problem that commonly arises in data centers. More specifically, we consider a model consisting of a fixed number of homogeneous servers denoted by $m\in\mathbb{N}$ and a fixed number of time slots denoted by $T\in\mathbb{N}$. Each server possesses two power states, that is a machine is either powered on (\textit{active state}) or powered off (\textit{sleep state}).
	
For any time slot $t\in[T]$, we have a \textit{mean arrival rate} denoted by $\lambda_t$, describing the amount of expected load to process in time slot $t$. The arrival rates are expected to be normalized such that each server can handle a load between 0 and 1 in any time slot. We denote the assigned load for server $i$ in time slot $t$ by $\lambda_{i,t}\in[0,1]$. Consequently, for any time slot $t$, we expect an arrival rate between 0 and $m$, that is $\lambda_t\in[0,m]$; otherwise, the servers would not be able to process the given load in time.

The costs incurred by a single machine are described by the sum of the machine's \textit{operating costs}, specified by \makebox{$f:[0,1]\rightarrow \mathbb{R}$}, as well as its \textit{(power state) switching costs}, specified by $\beta\in\mathbb{R}_{\ge 0}$. 
The operating costs function $f$ may not exclusively account for energy costs. For example, $f$ may also allow for costs incurred by delays, such as lost revenue caused by users waiting for their responses. Similarly, $\beta$ may also allow for delay costs, wear and tear costs or the like.\parencite{dyn-right-sizing}

We assume that a sleeping server does not cause any costs. Note that $f(0)$ denotes the costs incurred by an idle server, not a sleeping one; in particular, $f(0)$ may be non-zero. Further, we expect $f$ to be a convex function. This may seem like a notable restriction at first, but it indeed captures the behavior of most modern server models. Since we are dealing with homogeneous servers, $f$ and $\beta$ are the same for all machines.

For convenience, we assume that all machines sleep at time $t=0$ and also force all machines to sleep after the scheduling process, that is at times $t>T$. Consequently, any server must power down exactly as many times as it powers on. This allows us to consolidate a machine's power-up and power-down costs into $\beta$ and to model both costs as being incurred when powering up a server; that is, a model with power-up costs $\beta_\uparrow$ and power-down costs $\beta_\downarrow$ can be simply transferred to our model by setting $\beta\coloneqq \beta_\uparrow'\coloneqq\beta_\uparrow+\beta_\downarrow$ and $\beta_\downarrow'\coloneqq0$. As a consequence, we expect that there are no loads at times $t\notin[T]$, that is $\lambda_t=\lambda_{i,t}=0$ for $t\notin[T]$.

\section{Problem Statement}
Using the above definitions, we can define the input of our model by setting $\inp\coloneqq(m,T,\Lambda,\beta,f)$ where $\Lambda=(\lambda_1,\ldots,\lambda_T)$ is the sequence of arrival rates. We will subsequently identify a problem instance by its input $\inp$. Naturally, given a problem instance $\inp$, we want to schedule our servers in such a way that we minimize the sum of incurred costs while ensuring that we process the given loads in time. For this, consider for each server $i\in[m]$ the sequence of its states $S_i$ and the sequence of its assigned loads $L_i$:
\begin{align*}
	S_i&\coloneqq(s_{i,1},\ldots,s_{i,T})\in\{0,1\}^T\\
	L_i&\coloneqq(\lambda_{i,1},\ldots,\lambda_{i,T})\in[0,1]^T
\end{align*}
where $s_{i,t}\in\{0,1\}$ denotes whether server $i$ at time $t$ is sleeping (0) or active (1). Recall that we assume all machines are sleeping at times $t\notin[T]$; thus, for $t\notin[T]$ and $i\in[m]$, we have $s_{i,t}=0$. We can now define the sequence of all state changes and the sequence of all assigned loads:
\begin{align*}
	\mathcal{S}&\coloneqq(S_1,\ldots,S_m)\\
	\mathcal{L}&\coloneqq(L_1,\ldots,L_m)
\end{align*}
We will subsequently call a pair $\Sigma\coloneqq(\mathcal{S},\mathcal{L})$ a \textit{schedule}. Finally, we are ready to define our problem statement. Given an input $\inp$, our goal is to find a schedule $\Sigma$ that satisfies the following optimization:
\begin{align}
	&\text{minimize}&&\costs(\Sigma)\coloneqq\underbrace{\sum\limits_{t=1}^{T}\sum\limits_{i=1}^{m}\bigl(f(\lambda_{i,t}) s_{i,t}\bigr)}_{\text{operating costs}}+\underbrace{\beta\sum\limits_{t=1}^{T}\sum\limits_{i=1}^{m}\min\{0,s_{i,t}-s_{i,t-1}\}}_{\text{switching costs}}\label{eq:schedule_costs}\\ 
	&\text{subject to}&&\sum\limits_{i=1}^{m}(\lambda_{i,t} s_{i,t})=\lambda_t,\quad \forall t\in[T]\label{eq:feasible_constraint}
\end{align}
We call a schedule \textit{feasible} if it satisfies~\eqref{eq:feasible_constraint}, and \textit{optimal} if it satisfies~\eqref{eq:schedule_costs} and~\eqref{eq:feasible_constraint}.
