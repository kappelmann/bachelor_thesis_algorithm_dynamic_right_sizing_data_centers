% !TeX root = ../main.tex
% Add the above to each chapter to make compiling the PDF easier in some editors.

\chapter{Introduction}
\section{Motivation}
Data centers are a centerpiece to deal with the IT industry's fast-growing demand for data collection, processing, and storage~\parencite{cisco} and thus become an ever-increasing part of modern companies' infrastructure and thereby budget. One major and increasingly more important part of data centers' budget are energy costs~\parencite{hamilton}. The question of how to reduce these costs is thus of great interest.

One large fraction of data centers' energy costs is due to modern servers' disproportionate energy consumption to varying utilization levels, i.e.\ assigned loads. Even energy-efficient servers still consume about half their full power when nearly idle~\parencite{barroso}, wasting valuable energy while doing virtually nothing. One possible solution for this waste of energy, which we will address in this thesis, is to dynamically right-size a data center, that is dynamically adjusting the number of active servers and distributing the current workload by software. Algorithms dealing with this dynamic right-sizing thus face a scheduling problem: they have to decide when to power on machines or move machines to a power-saving mode (e.g.~sleep or shut down) during periods of high or low load, respectively; and further, they need to efficiently distribute the workload to the selected number of active servers.

It is common to consider two different variants of such optimization problems. The first variant, called \emph{offline optimization}, assumes that the whole problem data is given from the very beginning. The second variant, called \emph{online optimization}, processes its input piece-by-piece in a serial fashion, without having knowledge about the entire problem data from the very start. Algorithms dealing with these variants are consequently called offline and online algorithms, respectively.

We are not the first to address the described scheduling issue. Progress has been made in terms of offline as well as online variants for data center models with a homogeneous collection of servers. Lin et al.~\parencite{dyn-right-sizing} showed that the scheduling problem for this homogeneous case can be modeled as a convex optimization problem, and that the optimal offline solution satisfies a backward recurrence relation. This observation is used to develop a 3-competitive online algorithm based on an analogous forward recurrence relation for which progressively larger convex programs have to be solved at each time step. Bansal et al.~\parencite{bansal-soco} provided a different online approach, described in fairly mathematical terminology, which improved the upper bound on the optimal competitive ratio from 3 to 2 while also using a more general model, thereby improving applicability. 

All named solutions, however, give rise to two concerns: Firstly, they do not enforce the number of active servers to be integer, which means that their calculated solution does not directly translate to a real-world schedule. Secondly, both works do not provide procedures that can be conveniently implemented as a computer algorithm, but they settle for abstract mathematical procedures.

The goal of this thesis is thus to establish and verify new procedures that guarantee integer solutions and can be conveniently implemented as a computer algorithm. Our main results, presented in Chapter~\ref{chap:opt_offline_schedule} and Chapter~\ref{chap:approx_offline_scheduling}, are an optimal offline algorithm with pseudo-linear time complexity and a $(1+\beps)$-optimal offline algorithm with linear time complexity.

\section{Thesis Outline}
As a first step, we give a formal definition of the regarded data center model and its associated scheduling problem in Chapter~\ref{chap:model_descr_problem_state}. The definitions will be given in a fairly general form such that they apply to a majority of real-world data centers. We then proceed to examine the described model in Chapter~\ref{chap:preliminaries} in order to find useful patterns and properties, which will be used to greatly simplify the problem statement, thereby facilitating the subsequent main work of the thesis.

The main work is divided into two chapters: one dealing with the calculation of optimal solutions, and one focusing on approximative methods. The former, Chapter~\ref{chap:opt_offline_schedule}, begins with a reduction of the data centers' scheduling problem to the shortest path problem of an acyclic graph. This shortest path problem can then be solved in pseudo-polynomial-time, and its solution can be used to extract an optimal scheduling strategy for the data center. This first approach is further refined to derive an improved algorithm with pseudo-linear time complexity capable of calculating optimal scheduling strategies. 

To further minimize the algorithm's time complexity, we consider approximative solution strategies in Chapter~\ref{chap:approx_offline_scheduling}. For this, we modify the optimal algorithm's underlying graph such that its shortest path can be calculated in linear time. As a trade-off, the extracted schedule will be an approximation rather than an optimal solution. The first section of Chapter~\ref{chap:approx_offline_scheduling} deals with the general idea of this approximative approach and results in a 2-optimal algorithm, that is an algorithm whose calculated solution's cost is at most twice as much as an optimal solution's cost. The second section then generalizes this initial approach to derive a $(1+\beps)$-optimal algorithm with linear time complexity that is able to approximate optimal solutions with arbitrary precision.

The thesis finishes with a brief summary of the achieved results as well as some final thoughts about possible future developments and applications of the derived algorithms.
