% !TeX root = ../main.tex
% Add the above to each chapter to make compiling the PDF easier in some editors.

\chapter{Introduction}
TODO: Hardware prices vs.\ energy costs in data centers, related work and purpose of this paper (offline algorithm, approximation algorithm,\ldots).
\section{Motivation}

\section{Thesis Outline}
As a first step, we give a formal definition of the regarded data center model and its associated scheduling problem in Chapter~\ref{chap:model_descr_problem_state}. The definitions will be given in a fairly general form such that they apply to a majority of real-world data centers. We then proceed to examine the described model in Chapter~\ref{chap:preliminaries} in order to find useful patterns and properties, which will be used to greatly simplify the problem statement, thereby facilitating the subsequent main work of the thesis.

The main work is divided into two chapters: one dealing with the calculation of optimal solutions, and one focusing on approximative methods. The former, Chapter~\ref{chap:opt_offline_schedule}, begins with a reduction of the data centers's scheduling problem to a shortest path problem of an acyclic graph. This shortest path problem can then be solved in pseudo-polynomial time, and its solution can be used to extract an optimal scheduling strategy for the data center. This first approach is then refined to derive an improved algorithm with pseudo-linear time complexity capable of calculating optimal scheduling strategies. 

To further minimize the algorithm's time complexity, we consider approximative solution strategies in Chapter~\ref{chap:approx_offline_scheduling}. For this, we modify the optimal algorithm's underlying graph such that its shortest path can be calculated in linear time. As a tradeoff, the extracted schedule will be an approximation rather than an optimal solution. The first section of Chapter~\ref{chap:approx_offline_scheduling} will deal with the general idea of this approximative approach and result in an 2-optimal algorithm. The second section then generalizes this initial approach to derive an $(1+\beps)$-optimal algorithm with linear time complexity that is able to approximate optimal solutions with arbitrary precision.

The thesis finishes with a brief summary and assessment of the achieved results as well as some final thoughts about possible future developments and applications.
